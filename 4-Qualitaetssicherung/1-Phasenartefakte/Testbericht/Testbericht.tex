\documentclass[parskip=full]{scrartcl}

\usepackage[utf8]{inputenc}			% Umlaute, Sonderzeichen
\usepackage[ngerman]{babel}			% deutsche Sprache
\usepackage{enumitem}				% Listen
\usepackage{graphicx}				% Grafiken
\usepackage{hyperref}				% Hyperlinks
\usepackage[nonumberlist]{glossaries}		% Glossar
\usepackage{amsmath}
\usepackage{pdfpages}				% PDF einbinden


% Hurenkinder und Schusterjungen verhindern
\clubpenalty10000
\widowpenalty10000
\displaywidowpenalty=10000

\DeclareRobustCommand{\glossfirstformat}[1]{\textit{#1}}	% der erste Verweis im Dokument auf ...
\renewcommand*{\glsdisplayfirst}[4]{\glossfirstformat{#1#4}}	% ... einen Glossarbegriff wird kursiv markiert

% Kriterien sollen nicht kursiv erscheinen
\makeatletter
\renewcommand{\@begintheorem}[2]{\trivlist
	\item[\hskip \labelsep{\bfseries #1\ #2}]}
\makeatother


\makenoidxglossaries

\newglossaryentry{RasPi}{
	name=Raspberry Pi,
	plural=Raspberry Pis,
	description={Der Raspberry Pi ist ein Einplatinencomputer. In diesem Projekt dient der Raspberry Pi als Hardwareplattform, um Messwerte aus angeschlossenen Sensoren auszulesen}
}

\newglossaryentry{PhyPiDAQ}{
	name=PhyPiDAQ,
	description={PhyPiDAQ ist ein Framework zur Erfassung und Analyse von Messwerten mit einem Raspberry Pi. Siehe auch Abschnitt 4.3 ,,PhyPiDAQ`` sowie \url{https://github.com/GuenterQuast/PhyPiDAQ}}
}

\newglossaryentry{Science Labs}{
	name=Science Labs,
	description={Ein Science Lab ist ein Arbeitsplatz, welcher Schülern ermöglicht, wissenschaftliche Forschungen unter kontrollierten Bedingungen durchzuführen} 
}

\newglossaryentry{opensource}{
	name=Open Source,
	description={Software, deren Quelltext öffentlich eingesehen eingesehen werden kann, wird als ,,Open Source`` bzw. ,,quelloffen`` bezeichnet} 
}

\newglossaryentry{osl2}{
	name=OSL\textsuperscript{2},
	description={Open-Source-Lehrsoftware-Labor, siehe \url{https://formal.iti.kit.edu/projects/oslsl/?lang=de}}
}

\newglossaryentry{dragdrop}{
	name=Drag and Drop,
	description={Methode, um mit grafischen Benutzeroberflächen zu interagieren. Dabei wird ein Objekt erst mit der Maus festgehalten und an einen anderen Ort gezogen. Durch das Lösen der Maustaste wird das Objekt platziert}
}

\newglossaryentry{click}{
	name=Click,
	description={Betätigen der linken Maustaste}
}

\newglossaryentry{transformation}{
	name=Transformation,
	plural={Transformationen},
	description={Bausteine vom Typ Transformation haben einen oder mehrere Eingänge sowie einen oder mehrere Ausgänge. Für jeden Ausgang kann  ein Transformationsbaustein eine Vorschrift zur Berechnung eines Ausgangswertes aus einem Satz von Eingangswerten beinhalten. Eine Berechnungsvorschrift soll durch eine mathematische bzw. logische Funktionen oder durch eine programmtechnisch definierte Verarbeitung definiert werden können}
}

\newglossaryentry{konfigdata}{
	name=Konfigurationsdatei,
	plural=Konfigurationsdateien,
	description={Können das Messverhalten anpassen, beispielsweise die Anzahl der Messungen pro Zeiteinheit. Für jeden Sensor gibt es eine eigene Konfigurationsdatei}
}

\newglossaryentry{sensor}{
	name=Sensor,
	plural={Sensoren},
	description={Der Begriff ,,Sensor`` bezeichnet ein technisches Bauteil, welches physikalische Größen misst und analoge oder digitale Messwerte liefert. In unserer Anwendung werden Sensoren abstrakt als grafische Bausteine eines Messkonfiguration präsentiert. Ein solcher (logischer) Sensorbaustein muss alle Informationen referenzieren können, die zum Ansprechen eines tatsächlichen Sensors benötigt werden. Da ein Messgerät Ausgänge bzw. Messkanäle haben kann, muss ein Sensorbaustein mindestens einen oder auch mehrere Ausgänge haben. Eingänge besitzt ein Sensorbaustein nicht}
}

\newglossaryentry{messdaten}{
	name=Messdaten,
	description={Daten, welche die Anwendung von einem Sensor (über PhyPiDAQ-Schnittstelle) oder direkt aus einer Datei erhält}
}

\newglossaryentry{darstellung}{
	name=Darstellung,
	plural={Darstellungen},
	description={Bausteine vom Typ Darstellung haben einen oder mehrere Eingänge. Ein Darstellungsbaustein soll definieren können, wie ein Satz von Eingangswerten die Erstellung bzw. Aktualisierung einer Darstellung beeinflusst. Ausgänge besitzt ein Darstellungsbaustein nicht}
}

\newglossaryentry{python3}{
	name=Python 3,
	description={Python ist eine Skriptsprache, die auf dem Raspberry Pi als Standardsprache zur Programmierung vorgesehen ist. Python wurde zur Implementierung von PhyPiDAQ verwendet}
}

\newglossaryentry{JVM}{
	name={Java Virtual Machine},
	description={Die Java Virtual Machine (JVM) ist eine Plattform für die Ausführung von Java-Software, die von der Firma Oracle für alle gängigen Betriebssysteme bereitgestellt wird}
}

\newglossaryentry{DSGVO}{
	name=DSGVO,
	first={Datenschutz-Grundverordnung (DSGVO)},
	description ={Datenschutz-Grundverordnung der Europäischen Union vom 25. Mai 2018}
}

\newglossaryentry{Musskriterien}{
	name=Musskriterien,
	description ={Werden zusammen mit Soll- und Wunschkriterien bei der Abnahme eines Softwareprodukts überprüft und haben während der Entwicklung höchste Priorität. 
	Dass ein Musskriterium in den nachfolgenden Projektphasen nicht umgesetzt wird, ist nur dann zulässig, falls unerwartet unausweichliche Probleme bei der Umsetzung auftreten. 
	In diesem Fall ist es erforderlich, dass diese Probleme sehr genau dokumentiert werden}
}

\newglossaryentry{Sollkriterien}{
	name=Sollkriterien,
	description ={Werden zusammen mit Muss- und Wunschkriterien bei der Abnahme eines Softwareprodukts überprüft und haben während der Entwicklung mittlere Priorität. 
	Falls ein Sollkriterium umgesetzt werden kann, dann muss es nach Möglichkeit auch realisiert werden. 
	Falls ein Sollkriterium in den nachfolgenden Projektphasen nicht umgesetzt werden kann, so muss dies dokumentiert und begründet werden}
}

\newglossaryentry{Wunschkriterien}{
	name=Wunschkriterien,
	description ={Werden zusammen mit Muss- und Sollkriterien bei der Abnahme eines Softwareprodukts überprüft und haben während der Entwicklung eine niedrige Priorität. 
	Je nach Resourcenlage können sie nach Bearbeitung aller Muss- und Kannkriterien umgesetzt werden. 
	Falls ein Wunschkriterium nicht umgesetzt wird, so muss dies nicht begründet werden}
}

\newglossaryentry{Abgrenzungskriterien}{
	name=Abgrenzungskriterien,
	description ={Abgrenzungskriterien beschreiben Aspekte, die explizit nicht umgesetzt werden sollen}
}

\newglossaryentry{BenOber}{
	name=Benutzeroberfläche,
	description = {Steht für die Oberfläche, die der Benutzer verwendet um die Anwendung zu bedienen}
}

\newglossaryentry{UI}{
	name=UI,
	description={engl. User Interface; siehe: Benutzeroberfläche}
}

\newglossaryentry{GrafBenOber}{
	name=grafische Benutzeroberfläche,
	description = {Eine Benutzungsschnittstelle, die eine Anwendung durch Fenster, grafische Symbole, Menüs und Mauszeiger bedienbar macht}
}

\newglossaryentry{GUI}{
	name=GUI,
	description={engl. Graphical User Interface; siehe: Grafische Benutzeroberfläche},
}

\newglossaryentry{Konfigurationsbaustein}{
	name=Konfigurationsbaustein,
	plural=Konfigurationsbausteine,
	description ={Teil einer Messkonfiguration, der eine Teilaufgabe bestimmten Typs erfüllen kann. Es gibt Sensorbausteine, Konfigurationsbausteine und Darstellungsbausteine. Liegt am Ausgang eines Bausteins ein Wert an, so kann dieser an den Eingang eines nachgelagerten Bausteins weitergeleitet werden}
}

\newglossaryentry{Benutzerkonfiguration}{
	name =Messkonfiguration,
	plural=Messkonfigurationen,
	description ={Gerichteter zyklenfreier Graph mit Knoten vom Typ Sensor, Transformation oder Darstellung. Hierbei ist zu beachten, dass Sensoren keine Eingangskanten und Darstellungen keine Ausgangskanten haben dürfen}
}

\newglossaryentry{Bausteinprototyp}{
	name=Bausteinprototyp,
	plural=Bausteinprototypen,
	description={Baustein, von dem eine Kopie angelegt wird, wenn der Benutzer ein neues Baustein-Exemplar einem Entwurf hinzufügen möchte}
}

\newglossaryentry{Messlauf}{
	name=Messlauf,
	plural=Messläufe,
	description={Zeitabschnitt, in dem zu definierten Zeitpunkten an allen Bausteinen eines Entwurfs sukzessive die Werte an allen Ausgängen und Eingängen bestimmt werden}
}

\newglossaryentry{Stand-Alone-Kommunikation}{
	name={Stand-Alone-Kommunikation},
	description={Bezeichnet im Kontext unseres Software-Projekt die systeminterne Kommunikation innerhalb eines Betriebssystems, beispielsweise per Inter-Prozess-Kommunikation (IPC)}
	}


\newglossaryentry{Local-Loop}{
	name={Local-Loop},
	description={Bezeichnet einen virtuellen Netzwerkadapter, der Pakete, die durch ihn verschickt werden, unmittelbar danach auch wieder empfängt.}
}

\newglossaryentry{Model-View-Controller}{
	name={Model-View-Controller},
	description={Architekturmuster, dass die Software in die drei Komponenten: Model, View und Controller unterteilt. Dadurch sollen die einzelnen Komponenten unabhängig von einander verändert werden können.}
}

\newglossaryentry{JHotDraw}{
	name={JHotDraw},
	description={JHotDraw ist ein Open-Source, Java-basiertes Framework zur Erstellung von grafischen Editoren. Durch die einfachere Unterstützung von Drag-and-Dop, als komplexere Frameworks eine gute Alternative}
}

\newglossaryentry{allgemeinen Bausteinprototyp-Informationen}{
	name = {allgemeinen Bausteinprototyp-Informationen},
	description = {Dazu gehören folgende Eigenschaften: Name, Typ, Configurations ID, Initialisierungs ID, Nutzerinformationen (Hilfetext)}
}

\newglossaryentry{SWT}{
	name = {Standard Widget Toolkit},
	description = {Ein GUI Framework}
}

\newglossaryentry{GEF}{
	name = {Graphical Editing Framework},
	description = {Ein GUI Framework}
}

\newglossaryentry{Swing}{
	name = {Swing},
	description = {Ein GUI Toolkit}
}

\newglossaryentry{AWT}{
	name = {Abstract Window Toolkit},
	description = {Ein GUI Toolkit}
}

\newglossaryentry{Presenter}{
    name = {Presenter},
    description = {Presenter bezieht sich auf die gleichnahmige Komponente in einem Model-View-Presenter.}
}

\newglossaryentry{JaCoCo}{
    name = {JaCoCo},
    description = {JaCoCo ist eine freie Code-Überdeckungs Bibliothek für Java. Hier verwendete Version: 0.8.4}
}


\newglossaryentry{EclEmma}{
    name = {EclEmma},
    description = {EclEmma ist ein Plug-In für Eclipse für Code-Überdeckungsanalysen. Es basiert auf JaCoCo. Die hier verwendete Version ist 3.1.2}
}


\subject{Testbericht der Qualitätssicherungsphase}
\title{Definition und Durchführung~von Messwertverarbeitung für~den~Physikunterricht auf~Basis~eines~Raspberry~Pis}
\subtitle{Version 0.0.1}
\author{David Gawron \and Stefan Geretschläger \and Leon Huck \and Jan Küblbeck \and Linus Ruhnke}
\date{\today}


\begin{document}

\maketitle

\clearpage
\tableofcontents 					% generate pdf twice to update

\clearpage
\section{Ziel des Testberichts} \label{einleitung}

Das Ziel des Testberichtes ist es dem Leser einen Überblick über die verwendeten Testverfahren zu geben und die während der Qualitätssicherungsphase entdeckten Fehler zu dokumentieren. Die Qualitätssicherungsphase hat das Ziel, möglichst viele Fehler aufzudecken, diese zu korregieren und zu dokumentieren. Zusätzlich soll das unbemerkte Wiederauftreten bereits gefundener Fehler durch Regressionstests verhindert werden. Dabei werden die Funktionalitäten und deren Qualitäten getestet.


\subsection{Bedingungsüberdeckung}
Wir streben eine mehrfache Bedingungsüberdeckung an. Dadurch werden Zweig- ,Anweisungs- , einfache und minimal-mehrfache Bedingungsüberdeckung subsumiert. Eine einfache Bedingungsüberdeckung ist subsumiert nicht einmal die Anweisungsüberdeckung und ist somit ungeeignet. Eine minimal-mehrfache Bedingungsüberdeckung wäre ein guter Kompromiss zwischen Aufwand und Nutzen, allerdings verwendet unser Plug-In \gls{EclEmma} für \gls{JaCoCo} standardmäßig mehrfache Bedingungsüberdeckung. Außerdem ist die Anzahl an Bedingungen in unserer Anwendung noch überschaubar.
Eine Pfadüberdeckung streben wir nicht an, da dessen Aufwand mit 2 hoch k skaliert, wobei k die Anzahl an Anweisungen ist. 

\clearpage
\section{Planung der Qualitätssicherungsphase} \label{planung}

Die Qualitätssicherungsphase wird in drei Meilensteine aufgeteilt, siehe dazu Abbildung \ref{sollplan}. Der erste Meilenstein wird erfüllt, wenn das Modul Model der Anwendung eine hohe Testüberdeckung erreicht. Dabei sollen alle Tests automatisch mit J-Unit ablaufen. Das Model ist die Basis, die alle anderen Module benutzen und auch diese verbindet. Deshalb ist die erste Priorität eine getestetes Modul, um komplexe Folgefehler für die anderen Module zu verhindern. 

Im zweiten Meilenstein werden alle anderen Module, außer der GUI, getestet. Auch hier erfolgt das Testen über automatische J-Unit Tests.

Die GUI ist ein Sonderfall beim Testen, da diese nur sehr begrenzt mit automatischen Tests getestet werden kann. Deshalb wird diese im dritten Meilenstein getestet. Der Dritte Meilenstein umfasst die GUI und auch das Testen der gesamten Anwendung. Die GUI wird hauptsächlich über Klickstrecken getestet. Die gesamte Anwendung wird durch Testszenarien aus dem Pflichtenheft geprüft. Weiter werden Qualitätsanforderungen der Anwendung durch verschiedene Tests geprüft. Schließlich wird die Leistung und auch die Hardware für die Anwendung getestet. 

\begin{figure}[htbp]
	\begin{center}
		\includegraphics[width = 8cm]{Grafiken/sollplan.png}
		\caption{Der Sollpan für die Qualtätssicherungsphase.}
		\label{sollplan}
	\end{center}
\end{figure}


TODO: Wie ist der Plan am Ende der Phase aufgegangen?


\clearpage
\section{Gefundene Fehler und deren Regressionstests} \label{regression}

Dieses Kapitel umfasst die Regressionstests für gefundene und behobene Fehler. Die Tests sind nach Modul und Klassen strukturiert. Jeder Regressionstest verweist auf ein Issue der verwendeten Bugtracking-Software (hier GitHub).

\subsection{Übersicht aller Issues}
In der Tabelle \ref{issueOverView} wird angezeigt, wo ein Issue aufgetreten ist, und was für eine Kategorie es hat. Das Issue wird dabei durch seine Nummer repräsentiert. Zu den roten Issues gibt es keine Regressionstests, das diese nicht behoben wurden.

\begin{table}[h]
\begin{tabular}{| p{3cm} | p{2,2cm} | p{2,2cm}  |p{2,2cm} |p{2,2cm} |p{2,2cm}|}
	\hline
	\textbf{Art des Issue vs Fundort} & \textbf{Null Pointer} & \textbf{Index out Of Bounds} & \textbf{Path related} & \textbf{fehlerhafte Funktion} & \textbf{Sonstige} \\ \hline
	\textbf{Backend}
	& 
	
	&
	
	&
	
	&
	
	&
	34, 36
	\\ \hline
	
	\textbf{Cache}
	& 
	
	&
		
	&
	&
	&
	\\ \hline
	
	\textbf{Controller}
	& 
	
	&

	&
	&
	&
	\\ \hline
	\textbf{Gui}
	& 
	
	&
	&
	
	
	&
	15
	&
	\\ \hline
	
	\textbf{Model}
	& 
	7, 8, 9, 10, 12, 13, 19, 27
	&
	11, 18
	&	
	
	&
	21, 35
	&
	33, 53
	\\ \hline
	
	\textbf{Fileservice und Main}
	& 
	47
	&
	
	&
	57
	&
	50
	&
		
	\\ \hline

	\textbf{Gesamtzahl}
	& 
	&
	&
	&
	&
	\\ \hline
	
\end{tabular}
\caption{Übersicht über alle Issues.}
\label{issueOverView}
\end{table}

\subsection{Model}
\subsubsection{Measurement Configuration}

\begin{description}
 

\item []\textbf{Issue Nr.7 in der Methode getInChan} 

\begin{itemize}
\item []\textbf{Fehlersymptom:} Unbehandelte NullPointer Exception bei Eingabe einer ungültigen Id.
\item []\textbf{Fehlerursache:} Prüfen nach NullPointer Exception fehlt.
\item []\textbf{Fehlerbehebung:} Eine Null Prüfung wurde implementiert.
\item []\textbf{Verantwortlicher:} David Gawron
\end{itemize}

\item []\textbf{Issue Nr.8 in der Methode getOutChan} 

\begin{itemize}
\item []\textbf{Fehlersymptom:} Unbehandelte NullPointer Exception bei Eingabe einer ungültigen Id.
\item []\textbf{Fehlerursache:} Prüfen nach NullPointer Exception fehlt.
\item []\textbf{Fehlerbehebung:} Eine Null Prüfung wurde implementiert.
\item []\textbf{Verantwortlicher:} David Gawron
\end{itemize}

\item []\textbf{Issue Nr.9 in der Methode addConnection} 

\begin{itemize}
\item []\textbf{Fehlersymptom:} Unbehandelte NullPointer Exception bei Eingabe einer ungültigen Id.
\item []\textbf{Fehlerursache:} Prüfen nach NullPointer Exception fehlt.
\item []\textbf{Fehlerbehebung:} Eine Null Prüfung wurde implementiert.
\item []\textbf{Verantwortlicher:} David Gawron
\end{itemize}

\item []\textbf{Issue Nr.10 in der Methode removeConnection } 

\begin{itemize}
\item []\textbf{Fehlersymptom:} Unbehandelte NullPointer Exception bei Eingabe einer ungültigen Id.
\item []\textbf{Fehlerursache:} Prüfen nach NullPointer Exception fehlt.
\item []\textbf{Fehlerbehebung:} Eine Null Prüfung wurde implementiert.
\item []\textbf{Verantwortlicher:} David Gawron
\end{itemize}

\item []\textbf{Issue Nr.11 in der Methode createInChannelList } 

\begin{itemize}
\item []\textbf{Fehlersymptom:} Auftreten einer Index Out Of Bounds Exception.
\item []\textbf{Fehlerursache:} Eine Prüfung, ob der Index groß genug ist, fehlt.
\item []\textbf{Fehlerbehebung:} Der Fehler wird abgefangen durch einen Vergleich der Anzahl der InChannel zwischen yaml-File und Prototypblock.
\item []\textbf{Verantwortlicher:} David Gawron
\end{itemize}

\item []\textbf{Issue Nr.12 in der Methode getOutChanPosi } 

\begin{itemize}
\item []\textbf{Fehlersymptom:} NullPointer Exception beim Laden einer Messkonfiguration mit ungültigen Block Id.
\item []\textbf{Fehlerursache:} Prüfen nach NullPointer Exception fehlt.
\item []\textbf{Fehlerbehebung:} Es wird nach Null geprüft. Dann ergab sich eine Folgefehler, der sich in der Methode  createLoadedConnections als eine Index Out Of Bounds Exception äußerte. Durch das Implementieren einer Methode checkBlockInitId, die prüft, ob eine geladene Id auch gültig ist, wurde der Folgefehler behoben.
\item []\textbf{Verantwortlicher:} David Gawron
\end{itemize}


\item []\textbf{Issue Nr.13 in der Methode createInChannelList } 

\begin{itemize}
\item []\textbf{Fehlersymptom:} NullPointer Exception bei ungültiger Messkonfiguration mit einer fehlenden BlockChannelliste.
\item []\textbf{Fehlerursache:} Prüfen nach NullPointer Exception fehlt.
\item []\textbf{Fehlerbehebung:} Eine Prüfung nach Null wurde hinzugefügt.
\item []\textbf{Verantwortlicher:} David Gawron
\end{itemize}


\item []\textbf{Issue Nr.18 in der Methode removeBlock} 

\begin{itemize}
\item []\textbf{Fehlersymptom:} Der Versuch einen nicht existierenden Block zu entfernen, resultiert in eines Index Out Of Bounds Exception.
\item []\textbf{Fehlerursache:} Der Index wurde nicht geprüft.
\item []\textbf{Fehlerbehebung:} Eine Prüfung des Indexes wurde hinzugefügt. Außerdem wurde der Rückgabewert der Methode von void zu boolean geändert.
\item []\textbf{Verantwortlicher:} David Gawron
\end{itemize}

\item []\textbf{Issue Nr.19 in der Methode removeBlock} 

\begin{itemize}
\item []\textbf{Fehlersymptom:} Der Versuch eine Konfiguration ohne eine Liste von Block Ids zu laden, führt zu einer Null Pointer Exception.
\item []\textbf{Fehlerursache:} Es wurde nicht nach Null geprüft.
\item []\textbf{Fehlerbehebung:} Die betreffende Zeile wurde in einen schon existierenden Null-Check verschoben.
\item []\textbf{Verantwortlicher:} David Gawron
\end{itemize}




\item []\textbf{Fehler Nr.35 in der Methode getInitId} 

\begin{itemize}
\item []\textbf{Fehlersymptom:} Die Methode funktionierte nicht richtig und gab immer NULL zurück.
\item []\textbf{Fehlerursache:} Der Zugriff auf die Blöcke in der Hasmap der Konfigurationsblöcke schlägt fehl.
\item []\textbf{Fehlerbehebung:} Die KonfigurationsId wird nun über die Blockliste der Messkonfiguration geholt.
\item []\textbf{Verantwortlicher:} David Gawron
\end{itemize}



\end{description}



\subsection{Cache}
\subsection{Backend}
\subsection{Controller}
\subsection{Fileservice und Main}
\subsection{GUI}

\clearpage
\section{Testen der GUI} \label{gui}


\subsection{Testen der GUI durch Klickstrecken}


\subsection{Monkey Testing}

\clearpage
\section{Testen der Qualität} \label{quali}



\subsection{Hallway Usability Testing}


\subsection{Testen der Qualität der Funktionalitäten}


\clearpage
\section{Durchführen der Testfälle aus dem Pflichtenheft} \label{testszenarien}

\subsection{\textbf{T010} Starten der Anwendung und Hilfe}

DISCLAIMER: Der Testfall wurde so nicht wirklich durchgeführt, da der Pfad zur Textdatei noch nicht richtig funktioniert. Siehe Issue Nr. 15 in Git-Hub. Der Testfall wurde so angelegt, wie er später aussehen könnte. Er dient lediglich dazu, frühzeitig Feedback zu erhalten.

\begin{table}[h]
\begin{tabular}{| p{4cm} | p{10cm} |}
	\hline
	\textbf{Strukturelement} & \textbf{Beschreibung} \\ \hline
	\textbf{Testfallnummer (Pflichtenheft)}
	& 
	T10
	\\ \hline
	
	\textbf{Testfallverweis}
	& 
	hat ein Testfall vom Pflichtenheft eine JUnit-Test-Datei mit ein oder mehreren Tests?
	\\ \hline
	
	\textbf{(optional) Subunittests}
	& 
	\\ \hline
	
	\textbf{Verantwortlicher Tester}
	& 
	David
	\\ \hline
	
	
	\textbf{Vorbedingung}
	& 
	Die Anwendung ist als fat-Jar-Datei auf dem Rechner vorhanden. Es läuft keine Instanz dieser Anwendung.
	\\ \hline
	
	\textbf{ Testziel}
	& 
	Zu Testen ist das Verhalten des Anwendung, wenn sie gestartet wird. Außerdem soll die Hilfe-Funktion der Anwendung getestet werden.
	\\ \hline
	
	
	\textbf{Beschreibung}
	& 
	Die Anwendung öffnet sich bei dem Öffnen der fat-Jar-Datei. Dabei öffnet sich das Hauptfenster, in dem keine Messkonfiguration zu sehen ist. Drückt man den Knopf für die Hilfe, öffnet sich das Hilfefenster mit Informationen über die Benutzung der Anwendung.
	\\ \hline
	
	\textbf{Erwartetes Ergebnis}
	& 
	Das Hauptfenster und das Hilfefenster öffnen sich wie gewollt.
	\\ \hline
			
	\textbf{Verhalten im Fehlerfall}
	& 
	Eine Fehlermeldung wird angezeigt, falls beim Pfad zur Textdatei für das Hilfefenster keine Datei gefunden wurde.
	\\ \hline
	
	\textbf{Nachbedingung}
	& 
	Das Hauptfenster der Anwendung ist geöffnet. Es wird von dem geöffneten Hilfe-Fenster teilweise überdeckt.
	\\ \hline
	
	
	\textbf{Getestete Anforderungen}
	& 
	\textbf{F010} erreiche GUI nach Start, \textbf{F140} leere Darstellung nach Anwendungsstart, \textbf{F480} Hilfe zu Anwendung, \textbf{F490} Texte der Anwendung auf Deutsch
	\\ \hline
	
	
	
\end{tabular}
\caption{Testfall T10 aus dem Plfichtenheft: Öffnen der Anwendung und Hilfe.}
\label{testfallT10}
\end{table}




\subsection{\textbf{T020} Starten der Demo}

\subsection{\textbf{T030} Lehrer erstellt und speichert eine Messkonfiguration}

\subsection{\textbf{T040} Schüler bearbeitet Aufgabe}

\subsection{\textbf{T050} Schüler startet Messung und speichert Ergebnisse}

\subsection{\textbf{T200} Laden einer ungültigen Datei als Messkonfiguration}

\subsection{\textbf{T210} Starten einer ungültigen Messkonfiguration}

\subsection{\textbf{T220} Entfernen eines Sensors bei laufender Messung}

\clearpage
\section{Hardware Tests und sonstige Tests} \label{sonstiges}


\subsection{Leistung und Speicherverbrauch}


\subsection{Hardware Test der Sensoren}


\subsection{Testen auf verschiedenen Systemen}

\clearpage
\section{Glossar}\label{glossar}

\renewcommand*{\glossarysection}[2][]{}	% prevents double glossary section heading
\printnoidxglossaries				% generate pdf twice when adding new entries

\end{document}\grid
