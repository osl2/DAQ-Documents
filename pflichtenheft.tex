\documentclass[parskip=full]{scrartcl}

\usepackage[utf8]{inputenc}			% Umlaute, Sonderzeichen
\usepackage[ngerman]{babel}			% deutsche Sprache
\usepackage{enumitem}				% Listen
\usepackage{graphicx}				% Grafiken
\usepackage{hyperref}				% Hyperlinks
\usepackage[nonumberlist]{glossaries}		% Glossar
\usepackage{amsmath}

% Hurenkinder und Schusterjungen verhindern
\clubpenalty10000
\widowpenalty10000
\displaywidowpenalty=10000

\DeclareRobustCommand{\glossfirstformat}[1]{\textit{#1}}	% der erste Verweis im Dokument auf ...
\renewcommand*{\glsdisplayfirst}[4]{\glossfirstformat{#1#4}}	% ... einen Glossarbegriff wird kursiv markiert

% Kriterien sollen nicht kursiv erscheinen
\makeatletter
\renewcommand{\@begintheorem}[2]{\trivlist
	\item[\hskip \labelsep{\bfseries #1\ #2}]}
\makeatother


\makenoidxglossaries

\newglossaryentry{RasPi}{
	name=Raspberry Pi,
	plural=Raspberry Pis,
	description={Der Raspberry Pi ist ein Einplatinencomputer. In diesem Projekt dient der Raspberry Pi als Hardwareplattform, um Messwerte aus angeschlossenen Sensoren auszulesen}
}

\newglossaryentry{PhyPiDAQ}{
	name=PhyPiDAQ,
	description={PhyPiDAQ ist ein Framework zur Erfassung und Analyse von Messwerten mit einem Raspberry Pi. Siehe auch Abschnitt 4.3 ,,PhyPiDAQ`` sowie \url{https://github.com/GuenterQuast/PhyPiDAQ}}
}

\newglossaryentry{Science Labs}{
	name=Science Labs,
	description={Ein Science Lab ist ein Arbeitsplatz, welcher Schülern ermöglicht, wissenschaftliche Forschungen unter kontrollierten Bedingungen durchzuführen} 
}

\newglossaryentry{opensource}{
	name=Open Source,
	description={Software, deren Quelltext öffentlich eingesehen eingesehen werden kann, wird als ,,Open Source`` bzw. ,,quelloffen`` bezeichnet} 
}

\newglossaryentry{osl2}{
	name=OSL\textsuperscript{2},
	description={Open-Source-Lehrsoftware-Labor, siehe \url{https://formal.iti.kit.edu/projects/oslsl/?lang=de}}
}

\newglossaryentry{dragdrop}{
	name=Drag and Drop,
	description={Methode, um mit grafischen Benutzeroberflächen zu interagieren. Dabei wird ein Objekt erst mit der Maus festgehalten und an einen anderen Ort gezogen. Durch das Lösen der Maustaste wird das Objekt platziert}
}

\newglossaryentry{click}{
	name=Click,
	description={Betätigen der linken Maustaste}
}

\newglossaryentry{transformation}{
	name=Transformation,
	plural={Transformationen},
	description={Bausteine vom Typ Transformation haben einen oder mehrere Eingänge sowie einen oder mehrere Ausgänge. Für jeden Ausgang kann  ein Transformationsbaustein eine Vorschrift zur Berechnung eines Ausgangswertes aus einem Satz von Eingangswerten beinhalten. Eine Berechnungsvorschrift soll durch eine mathematische bzw. logische Funktionen oder durch eine programmtechnisch definierte Verarbeitung definiert werden können}
}

\newglossaryentry{konfigdata}{
	name=Konfigurationsdatei,
	plural=Konfigurationsdateien,
	description={Können das Messverhalten anpassen, beispielsweise die Anzahl der Messungen pro Zeiteinheit. Für jeden Sensor gibt es eine eigene Konfigurationsdatei}
}

\newglossaryentry{sensor}{
	name=Sensor,
	plural={Sensoren},
	description={Der Begriff ,,Sensor`` bezeichnet ein technisches Bauteil, welches physikalische Größen misst und analoge oder digitale Messwerte liefert. In unserer Anwendung werden Sensoren abstrakt als grafische Bausteine eines Messkonfiguration präsentiert. Ein solcher (logischer) Sensorbaustein muss alle Informationen referenzieren können, die zum Ansprechen eines tatsächlichen Sensors benötigt werden. Da ein Messgerät Ausgänge bzw. Messkanäle haben kann, muss ein Sensorbaustein mindestens einen oder auch mehrere Ausgänge haben. Eingänge besitzt ein Sensorbaustein nicht}
}

\newglossaryentry{messdaten}{
	name=Messdaten,
	description={Daten, welche die Anwendung von einem Sensor (über PhyPiDAQ-Schnittstelle) oder direkt aus einer Datei erhält}
}

\newglossaryentry{darstellung}{
	name=Darstellung,
	plural={Darstellungen},
	description={Bausteine vom Typ Darstellung haben einen oder mehrere Eingänge. Ein Darstellungsbaustein soll definieren können, wie ein Satz von Eingangswerten die Erstellung bzw. Aktualisierung einer Darstellung beeinflusst. Ausgänge besitzt ein Darstellungsbaustein nicht}
}

\newglossaryentry{python3}{
	name=Python 3,
	description={Python ist eine Skriptsprache, die auf dem Raspberry Pi als Standardsprache zur Programmierung vorgesehen ist. Python wurde zur Implementierung von PhyPiDAQ verwendet}
}

\newglossaryentry{JVM}{
	name={Java Virtual Machine},
	description={Die Java Virtual Machine (JVM) ist eine Plattform für die Ausführung von Java-Software, die von der Firma Oracle für alle gängigen Betriebssysteme bereitgestellt wird}
}

\newglossaryentry{DSGVO}{
	name=DSGVO,
	first={Datenschutz-Grundverordnung (DSGVO)},
	description ={Datenschutz-Grundverordnung der Europäischen Union vom 25. Mai 2018}
}

\newglossaryentry{Musskriterien}{
	name=Musskriterien,
	description ={Werden zusammen mit Soll- und Wunschkriterien bei der Abnahme eines Softwareprodukts überprüft und haben während der Entwicklung höchste Priorität. 
	Dass ein Musskriterium in den nachfolgenden Projektphasen nicht umgesetzt wird, ist nur dann zulässig, falls unerwartet unausweichliche Probleme bei der Umsetzung auftreten. 
	In diesem Fall ist es erforderlich, dass diese Probleme sehr genau dokumentiert werden}
}

\newglossaryentry{Sollkriterien}{
	name=Sollkriterien,
	description ={Werden zusammen mit Muss- und Wunschkriterien bei der Abnahme eines Softwareprodukts überprüft und haben während der Entwicklung mittlere Priorität. 
	Falls ein Sollkriterium umgesetzt werden kann, dann muss es nach Möglichkeit auch realisiert werden. 
	Falls ein Sollkriterium in den nachfolgenden Projektphasen nicht umgesetzt werden kann, so muss dies dokumentiert und begründet werden}
}

\newglossaryentry{Wunschkriterien}{
	name=Wunschkriterien,
	description ={Werden zusammen mit Muss- und Sollkriterien bei der Abnahme eines Softwareprodukts überprüft und haben während der Entwicklung eine niedrige Priorität. 
	Je nach Resourcenlage können sie nach Bearbeitung aller Muss- und Kannkriterien umgesetzt werden. 
	Falls ein Wunschkriterium nicht umgesetzt wird, so muss dies nicht begründet werden}
}

\newglossaryentry{Abgrenzungskriterien}{
	name=Abgrenzungskriterien,
	description ={Abgrenzungskriterien beschreiben Aspekte, die explizit nicht umgesetzt werden sollen.}
}

\newglossaryentry{BenOber}{
	name=Benutzeroberfläche,
	description = {Steht für die Oberfläche, die der Benutzer verwendet um die Anwendung zu bedienen}
}

\newglossaryentry{UI}{
	name=UI,
	description={engl. User Interface; siehe: Benutzeroberfläche}
}

\newglossaryentry{GrafBenOber}{
	name=grafische Benutzeroberfläche,
	description = {Eine Benutzungsschnittstelle, die eine Anwendung durch Fenster, grafische Symbole, Menüs und Mauszeiger bedienbar macht}
}

\newglossaryentry{GUI}{
	name=GUI,
	description={engl. Graphical User Interface; siehe: Grafische Benutzeroberfläche},
}

\newglossaryentry{Konfigurationsbaustein}{
	name=Konfigurationsbaustein,
	plural=Konfigurationsbausteine,
	description ={Teil einer Messkonfiguration, der eine Teilaufgabe bestimmten Typs erfüllen kann. Es gibt Sensorbausteine, Konfigurationsbausteine und Darstellungsbausteine. Liegt am Ausgang eines Bausteins ein Wert an, so kann dieser an den Eingang eines nachgelagerten Bausteins weitergeleitet werden}
}

\newglossaryentry{Benutzerkonfiguration}{
	name =Messkonfiguration,
	plural=Messkonfigurationen,
	description ={Gerichteter zyklenfreier Graph mit Knoten vom Typ Sensor, Transformation oder Darstellung. Hierbei ist zu beachten, dass Sensoren keine Eingangskanten und Darstellungen keine Ausgangskanten haben dürfen}
}

\newglossaryentry{Bausteinprototyp}{
	name=Bausteinprototyp,
	plural=Bausteinprototypen,
	description={Baustein, von dem eine Kopie angelegt wird, wenn der Benutzer ein neues Baustein-Exemplar einem Entwurf hinzufügen möchte}
}

\newglossaryentry{Messlauf}{
	name=Messlauf,
	plural=Messläufe,
	description={Zeitabschnitt, in dem zu definierten Zeitpunkten an allen Bausteinen eines Entwurfs sukzessive die Werte an allen Ausgängen und Eingängen bestimmt werden}
}

\newglossaryentry{Stand-Alone-Kommunikation}{
	name={Stand-Alone-Kommunikation},
	description={Bezeichnet im Kontext unseres Software-Projekt die systeminterne Kommunikation innerhalb eines Betriebssystems, beispielsweise per Inter-Prozess-Kommunikation (IPC)}
	}


\newglossaryentry{Local-Loop}{
	name={Local-Loop},
	description={Bezeichnet einen virtuellen Netzwerkadapter, der Pakete, die durch ihn verschickt werden, unmittelbar danach auch wieder empfängt.}
}

\subject{Pflichtenheft}
\title{Visuelle Programmiersprache für den Physikunterricht zur Datenerfassung auf einem Raspberry Pi}
\subtitle{Version 1.0.0}
\author{David Gawron \and Stefan Geretschläger \and Leon Huck \and Jan Küblbeck \and Linus Ruhnke}
\date{\today}


\begin{document}

\maketitle

\clearpage
\tableofcontents 					% generate pdf twice to update

\clearpage
\section{Produktübersicht} %Ein Abstrakt aus allen Sektionen

Die im Rahmen des \gls{osl2} zu entwickelnde Anwendung soll es Lehrern ermöglichen, Schülern die Grundlagen von physikalischen Messtechniken zu vermitteln (siehe Kapitel~\ref{zielbestimmung}). Die erhoffte Konsequenz ist, dass mehr Schüler für wissenschaftliche und technische Fächer motiviert werden.
Demzufolge kommt die Anwendung in Schulen zum Einsatz. 
Die Zielgruppe sind Schüler in der siebten Klasse aufwärts (siehe Kapitel \ref{produkteinsatz}).
Die Anwendung läuft in einer \gls{JVM} und erhält Messwerte von \glspl{sensor} (siehe Kapitel \ref{produktumgebung}). 
Die hardwarenahe Ansteuerung der Sensoren soll durch das bereits vorhandene Softwarepaket \gls{PhyPiDAQ} erfolgen (siehe Kapitel \ref{systemmodell}).
Detaillierte funktionale und nicht funktionale Anforderungen sind in Kapitel \ref{funktionale} bzw. \ref{nichtfunktionale} aufgelistet.
Die Anwendung soll über eine \gls{GrafBenOber}, mit Maus und Tastatur, bedient werden. (siehe Kapitel \ref{gui}).

\clearpage
\section{Zielbestimmung}\label{zielbestimmung}

Die Anwendung soll es Lehrern ermöglichen, Schülern ab der siebten Klasse Grundkenntnisse der digitalen Messwerterfassung in einer für Schüler interessanten und motivierenden Art und Weise näher zu bringen. 

In einem einfachen Versuchsaufbau sind die wesentlichen Schritte einer messtechnischen Verarbeitung die Folgenden: 

\begin{enumerate}
	\item Abtastung der physikalischen Größen
	
	\item Transformation der Messwerte durch mathematische Funktionen oder programmtechnisch definierte Verarbeitung
	
	\item die Darstellung der transformierten Werte in einer zur Beantwortung der Fragestellung geeigneten Form.
	
\end{enumerate}

Das projektierte Softwareprodukt soll die Schüler dabei unterstützen, die Schritte einer messtechnischen Verarbeitung spezifisch für eine physikalisches Experiment zu definieren und zueinander in Beziehung zu setzen. Hierzu soll das Produkt sogenannte \glspl{Konfigurationsbaustein} unterschiedlichen Typs anbieten, aus denen ein Schüler dann eine Messkonfiguration entwerfen kann. 

 Der grundlegende softwaretechnische Ansatz des projektierten Systems besteht darin, während eines Messvorgangs für ähnliche Teilaufgaben jeweils einen bestimmten Bausteintyp vorzusehen und im Umkehrschluss gänzlich unterschiedliche Teilaufgaben einer Messkonfiguration von Konfigurationsbausteinen unterschiedlichen Typs erledigen zu lassen.
 
  Eine Analyse zeigt, wie die angebotenen Konfigurationsbausteine geartet sein müssen, damit sie während eines Messvorgangs das ganze Spektrum der typischen Teilaufgaben erfüllen können:

\begin{itemize}
	
	\item Bausteine vom Typ Sensor haben nur Ausgänge. Ein solcher (logischer) Sensorbaustein muss alle Informationen referenzieren können, die zum Ansprechen eines Messgeräts bzw. physikalischen Sensors benötigt werden. Da ein Messgerät Ausgänge bzw. Messkanäle haben kann, kann ein Sensorbaustein mehrere Ausgänge haben. Eingänge besitzt ein Sensorbaustein nicht. 
	
	\item Bausteine vom Typ Transformation haben einen oder mehrere Eingänge sowie einen oder mehrere Ausgänge. Für jeden Ausgang kann  ein Transformationsbaustein eine Vorschrift zur Berechnung eines Ausgangswertes aus einem Satz von Eingangswerten beinhalten. Eine Berechnungsvorschrift soll durch eine mathematische Funktionen oder durch eine programmtechnisch definierte Verarbeitung definiert werden können.
	
	\item Bausteine vom Typ Darstellung haben einen oder mehrere Eingänge. Ein Darstellungsbaustein soll definieren können, wie ein Satz von Eingangswerten die Erstellung bzw. Aktualisierung einer Darstellung beeinflusst. Ausgänge besitzt ein Darstellungsbaustein nicht. 
	
\end{itemize}

Die Konfigurationsbausteine sollen anpassbar sein, nicht nur um die Software für möglichst viele physikalische Fragestellungen einsetzen zu können, sondern auch um den Bedürfnissen unterschiedlicher Altersstufen gerecht zu werden. 

Natürlich soll dann auch der Betrieb der Messkonfiguration gesteuert werden können und das Produkt sollte eventuell auch Einblicke in Abläufe während des Betriebs unterstützen, beispielsweise zur Analyse unerwünschten Verhaltens.

\subsection{Eignung für den Unterricht}

Aus didaktischer Sicht überflüssige technische Details wie z. B. die zum Auslesen der Sensoren notwendigen Protokolle sollen vor dem Schüler in jedem Fall verborgen bleiben.
Der Schüler wird so nicht überfordert, sondern soll ermutigt werden, selbstständig mit der Software umzugehen. 
Dabei kann er im spielerischen Umgang mit Versuchsaufbauten Prinzipien der digitalen Messtechnik wie z. B. Kaskadierung und natürlich auch das Grundprinzip von Ursache und Wirkung erfahren.

Es wird eine \gls{GrafBenOber} angeboten, die es dem Schüler ermöglicht, allein per \gls{dragdrop} eine Messkonfiguration zu entwerfen. 
Überflüssige Details, die dahinter stecken, bleiben vor dem Schüler verborgen.
Die Anwendung motiviert den Schüler dazu, mit \glspl{sensor}, \glspl{transformation} und \glspl{darstellung} zu spielen und Dinge auszuprobieren. 
Dabei wird ihm durch eine intuitive Status- und Fehleranzeige gezeigt, ob sein Entwurf funktioniert. 
Wenn nicht, dann hilft sie ihm das Problem einzugrenzen und gibt Hinweise zur Problembehebung. 
Eine lästige Fehlersuche soll dem Schüler weitestgehend erspart bleiben. 

Außerdem liefert die Anwendung dem Schüler zu den vorhandenen Bausteinen und zu der Anwendung allgemein die nötigen Informationen, die er für die Nutzung braucht. Dabei wird auf ein ausführliches Tutorial am Anfang verzichtet. Die Anwendung liefert die Informationen häppchenweise durch Informationsanzeigen an den jeweilig relevanten Stellen. Damit findet der Schüler die Hilfe, die er sucht, an der Stelle, an der er sie braucht.

Die Anwendung ermöglicht es dem Lehrer vor dem Unterricht, eine Reihe von Messversuchen teilweise oder ganz zu konfigurieren und zu speichern. Dabei kann er genau bestimmen, was er den Schülern zeigen will. Diese Konfigurationen kann er dann schnell und einfach im Unterricht zum Einsatz bringen. 

Weiter ermöglicht es die Anwendung, dass Schüler aus höheren Klassen mit mehr Details der digitalen Messwerterfassung zusammen gebracht werden. Diese nutzen keine oder nur eine abstrakt bzw. lückenhaft vorgefertigte Konfiguration. Sie können dann auch selbst einen Baustein modifizieren oder erstellen. Trotz des höheren Detailgrads bleibt die Anwendung übersichtlich und strukturiert. Damit bleiben auch komplexere Messversuche für den Schüler motivierend. 

Die Anwendung ist als Bindeglied zwischen \gls{PhyPiDAQ} und dem Nutzer zu verstehen. Sie ermöglicht dem Nutzer, über eine übersichtliche, strukturierte und intuitive \gls{GrafBenOber} sowie über eine einfache Bedienung die Nutzung eines \gls{RasPi} mit \gls{PhyPiDAQ}. Dadurch soll dem Schüler digitale Messwerterfassung, Physik und auch Informatik in einer motivierenden Weise näher gebracht werden. Womöglich kann die Anwendung den Schüler auch für diese Themen begeistern.  

\subsection{Begriffsdefinitionen}

Im Folgenden werden \gls{Musskriterien} (MK), \gls{Sollkriterien} (SK), \gls{Wunschkriterien} (WK) und \gls{Abgrenzungskriterien} (AK) zur Abnahme des projektierten Softwareprodukts definiert. Während der Entwicklung wird ihre Umsetzung mit hoher, mittlerer und niedriger Priorität betrieben.

\newtheorem{MussKrit}{MK}
\newtheorem{SollKrit}{SK}
\newtheorem{WunschKrit}{WK}


\subsection{Entwurf von Messkonfigurationen}

\begin{itemize}
	
	\item 
	\begin{MussKrit} 			
		Der Benutzer muss einen Baustein zur Messkonfiguration hinzufügen können, d. h. er soll einen Prototypen auswählen und das Bausteinexemplars auf dem Konfigurationsfeld platzieren können. 
	\end{MussKrit}
	
	\item 
	\begin{MussKrit} 		
		Der Benutzer muss die Baustein-Eigenschaften und die Position des Bausteins auf der Konfigurationsfeld nach seinen Wünschen anpassen können.
	\end{MussKrit}
	
	\item 
	\begin{MussKrit} 			
		Der Benutzer muss einen Baustein aus der Messkonfiguration entfernen können.
	\end{MussKrit}
	
	\item 
	\begin{MussKrit} 			
		Der Benutzer muss eine Weiterleitungsregel bzw. einen ,,Draht`` vom Ausgang eines Bausteins zum Eingang eines anderen Bausteins erstellen können.
	\end{MussKrit}
	
	\item 
	\begin{MussKrit} 			
		Der Benutzer muss eine Weiterleitungsregel bzw. einen ,,Draht`` löschen können.
	\end{MussKrit}
	
	\item 
	\begin{SollKrit} 			
		Das System soll dem Benutzer eine Undo- und eine Redo-Funktion anbieten.
	\end{SollKrit}
			
	\item 
	\begin{WunschKrit} 
		Hinzufügen, Darstellen, Bearbeiten und Löschen ergänzender Text- und evtl. auch Bildinformation, beispielsweise zur ausführlichen Beschreibung der physikalische Hintergründe oder für Memos, To-dos, etc.
	\end{WunschKrit}.
	
\end{itemize}

\subsection{Handhabung von Bausteinprototypen}

Möchte ein Benutzer einem Entwurf ein neues Bausteinexemplar hinzufügen, so wird einem \glspl{Bausteinprototyp} eine Kopie angelegt.

\begin{itemize}
	
	\item 
	\begin{SollKrit} 
		Der Benutzer soll die Eigenschaften eines Bausteinprototyps einsehen können.
	\end{SollKrit}
	
	\item Der Benutzer soll von allen Bausteintypen (Sensor, Transformation und Darstellung) neue Prototypen erstellen können. Dies kann auf zweierlei Art und Weise geschehen: 
	
	\begin{itemize}
		
		\item 
		\begin{SollKrit} 
			durch Anpassung bereits vorhandener (evtl. auch generischer) Bausteintypen,
		\end{SollKrit}
		
		\item 
		\begin{WunschKrit} 
			durch Erweiterung um neue Softwarekomponenten, die definierte programmtechnische Schnittstellen nutzen.
		\end{WunschKrit}
		
	\end{itemize}
	
	\item 
	\begin{SollKrit}
		Der Benutzer soll die Eigenschaften benutzererstellter Bausteinprototypen verändern können.
	\end{SollKrit}
	
	\item 
	\begin{SollKrit}
		Der Benutzer soll benutzererstellte Bausteinprototypen löschen können.
	\end{SollKrit}
	
	
\end{itemize}

\subsection{Gewährleisten von Persistenz}

\begin{itemize}
		
	\item 
	\begin{MussKrit} 			
		Eine Messkonfiguration, einschließlich der grafischen Anordnung der Bausteine muss sich in einer Datei speichern lassen.
	\end{MussKrit}
	
	\item 
	\begin{MussKrit} 			
		Eine zuvor gespeicherte Messkonfiguration muss sich zum weiteren Gebrauch durch Laden aus der entsprechenden Datei wiederherstellen lassen. 
	\end{MussKrit}
	
	\item 
	\begin{SollKrit} 
		Vom Benutzer erstellte Bausteinprototypen sollen sich auswählen und in einer Datei speichern lassen.			
	\end{SollKrit}
	
	\item 
	\begin{SollKrit} 			
		Zuvor gespeicherte Bausteinprototypen sollen sich zum weiteren Gebrauch durch Laden aus der entsprechenden Datei wiederherstellen lassen. 
	\end{SollKrit}
	
\end{itemize}

\subsection {Bereitstellung vorgefertigter Teile}

\begin{itemize}
	
	\item Die Anwendung enthält eine Reihe von vorgefertigten Bausteinprototypen: 
	\begin{itemize}
		
		\item 
		\begin{MussKrit} 		
			Sensorbausteine generischer Natur, beispielsweise für einen\newline Analog/Digital-Wandler, sowie bereits spezialisierte Sensorbausteine, beispielsweise für Licht, Temperatur oder Lage.	
		\end{MussKrit} 
		
		\item 
		\begin{MussKrit} 		
			Transformationsbausteine generischer Natur, beispielsweise für lineare Abbildungen oder für bedingte Ausgaben, sowie speziellere Transformationsbausteine, beispielsweise ein Baustein zur Datenstromduplizierung oder ein Multiplexer als beispielhafte Spezialisierung eines ,,bedingten`` Bausteins. 
				
		\end{MussKrit}
		
		\item 
		\begin{MussKrit} 	
			für die meisten Zwecke gut anpassbare Darstellungsbausteine, beispielsweise zur Generierung von Tabellen oder Koordinatensystemen.					
		\end{MussKrit}
	\end{itemize}
	
	\item 
	\begin{SollKrit} 
		Die Anwendung sollte mindestens drei vorgefertigte Messkonfigurationen unterschiedliche Komplexität enthalten. 			
	\end{SollKrit}

	\item
	\begin{WunschKrit}
Die Anwendung enthält Spiele, z.B kann ein Schüler versuchen ein vorgegebenes Messergebniss möglichst genau nachzumachen oder zu reproduzieren.
\end{WunschKrit}

	
\end{itemize}


\subsection{Handhabung von Messläufen}

Die folgenden Features bzw. Kriterien sollen den Benutzer dabei unterstützen, zufriedenstellende \glspl{Messlauf} zu erhalten.


\begin{itemize}
	
	\item 
	\begin{MussKrit} 			
		Der Benutzer muss einen Messlauf mit adäquater Eingabe starten, pausieren, fortführen und beenden können. 
	\end{MussKrit}
	
	\item 
	\begin{MussKrit} 	
		Der Benutzer muss vor Start eines Messlaufs die Zeitpunkte definieren können, zu denen an den Ausgängen der Sensorbausteine die zuletzt abgetasteten Messwerte registriert und von den nachgelagerten Bausteinen gemäß ihrer Definition sukzessive weiterverarbeitet werden.		
	\end{MussKrit}
	
	\item 
	\begin{SollKrit} 
		Während des Entwurfs bzw. der Bearbeitung einer Messkonfiguration soll der Nutzer Checks vornehmen, um überprüfen, ob die gesamte Messkonfiguration oder Teile in ihrem derzeitigen Zustand funktionieren. Bei negativem Ergebnis erhält der Benutzer möglichst aussagekräftige Hinweise. 
 
	
		\begin{itemize}
			
			\item Ist beispielsweise ein physikalischer Sensor nicht funktionsfähig, beispielsweise weil er nicht oder fehlerhaft angeschlossen ist, so kann der Benutzer dies frühzeitig erkennen.
			
			\item Sind Ein- oder Ausgänge von Bausteinen nicht verdrahtet, obwohl dies für  eine ordnungsgemäße Funktion des Bausteins erforderlich ist, so kann der Benutzer dies ebenso frühzeitig erkennen.
			
		\end{itemize}
	\end{SollKrit}
	
	
	\item 
	\begin{WunschKrit} 			
		Bausteine vom Typ Sensor sollen als Alternative zu abgetasteten Messwerten auch vorgehaltene Testdaten bereithalten können, so dass eine Messkonfiguration auch ohne Messgeräte oder physikalische Sensoren getestet oder vorgeführt werden kann.
	\end{WunschKrit}	
	
	\item 
	\begin{WunschKrit} 			
		Der Benutzer soll bei jedem Baustein den lokalen Datenfluss während eines Messlaufs aufzeichnen, einsehen und speichern können. 
	\end{WunschKrit}
		
	\item
	\begin{WunschKrit} 
		Für einen Darstellungsbaustein wäre eine Druckfunktion für die von ihr generierte Darstellung wünschenswert. 
	\end{WunschKrit}
	
	
\end{itemize}



\subsection {Benutzbarkeit der GUI}
\begin{itemize}
	
	\item 
	\begin{WunschKrit} 		
		Die \gls{GUI} bietet als Sprache mindestens Deutsch an und muss leicht um weitere Sprachen ergänzt werden können. Sobald die Anwendung mehr als eine Sprache anbietet, kann der Benutzer die GUI von einer Sprache auf eine andere umstellen.	
	\end{WunschKrit}
	\item 
	\begin{SollKrit}
		Die Anwendung reagiert zuverlässig auf Benutzereingaben.
	\end{SollKrit}
	\item 
	\begin{SollKrit} 
		Beim Entwurf einer Messkonfiguration soll der Benutzer die von ihm beabsichtigten Aktionen per Drag and Drop anstoßen können. Dies betrifft beispielsweise das Hinzufügen eines Konfigurationsbausteins oder das Erstellen einer Kante zwischen zwei Bausteinen.
	\end{SollKrit}
	\item 
	\begin{WunschKrit}
		Da nicht immer alle Eingänge oder Ausgänge Verwendung finden, sollen nicht benötigte Eingänge oder Ausgänge bei einem Baustein bzw. Bausteinprotoyp ausgeblendet bzw. unsichtbar gemacht werden können. 
	\end{WunschKrit}
	\item 
	\begin{SollKrit}
		Die Anwendung soll interaktiv zugängliche Erklärungen und Informationen zu den einzelnen GUI-Elementen bereit halten.
	\end{SollKrit}
	\item
	\begin{SollKrit}
		Die Anwendung soll, wo möglich, Fehlverhalten des Benutzers verhindern oder mit Hinweisen davor warnen. 
	\end{SollKrit}
	\item 
	\begin{WunschKrit} 
		Die Farben der GUI-Elemente sollen für den Benutzer anpassbar sein. Bei Benutzung des vorgeschlagenen Farbschemas soll die \gls{GUI} möglichst barrierefrei sein, d.h. die Semantik sollte auch für Benutzer mit Farberkennungsschwächen ohne Probleme zu erkennen sein.
	\end{WunschKrit} 
	
	\item 
	\begin{WunschKrit} 
		Zur weiteren Barrierefreiheit sollte das Produkt beispielsweise für Schüler mit Sehbehinderung variable Darstellungsgrößen anbieten.
	\end{WunschKrit}
	
	
\end{itemize}



\subsection{Abgrenzungskriterien}

\begin{itemize}
	
	\item \textbf{AK 1} Unterschiedliche Benutzerkonten bzw. personalisierte Arbeitsumgebungen  für die einzelnen Benutzer sieht das Produkt nicht vor. 
	
	\item \textbf{AK 2} Die zu erstellende Anwendung sieht keinerlei Erfassung oder Speicherung personenbezogener Nutzerdaten vor. Der Einsatz der Software erfordert dementsprechend keine Einverständniserklärung der Nutzer gemäß europäischer \gls{DSGVO}.
	
\end{itemize}

\clearpage
\section{Produkteinsatz}\label{produkteinsatz}

\subsection{Anwendungsbereiche}

Die Anwendung ist für die Verwendung in Schulklassen ab der 7. bis zu 10. Klasse für Schüler und Lehrer im Physikunterricht konzipiert. Ebenfalls soll die Anwendung in Physikprojekten, sowie \gls{Science Labs} verwendet werden können. 
Als \gls{opensource} Projekt im Rahmen des \gls{osl2} steht die Anwendung jedoch jedem Interessierten, zur Verwendung und Weiterentwicklung, zur Verfügung.


\subsection{Zielgruppe}

Die Zielgruppen sind hauptsächlich Schülerinnen und Schüler im Physikunterricht.  Diese verwenden die Anwendung, um erste eigene Messungen durchzuführen. Ein Ziel ist es, den Schülern erste Einblicke in Messtechniken zu geben. Weiter soll ihnen der Zusammenhang zwischen Ursache und Wirkung gezeigt werden. 
Einerseits soll die Anwendung physikbegeisterten Schüler ansprechen. Andererseits soll sie auch von Schülern, die kein großes Vorwissen in Physik und Messtechnik haben, verwendet werden können. Diesen Schülern soll ebenfalls eine interessante und einfach zu bedienende Plattform geboten werden.
Die Schüler brauchen ebenfalls keine Kenntnisse im Umgang mit der Programmiersprache Python oder mit einem Raspberry Pi.


\subsection{Betriebsbedingungen}

Die Anwendung läuft auf Computern, welche über eine Schnittstelle mit einem \gls{RasPi} verbunden sind, auf welchem das Programm \gls{PhyPiDAQ} läuft. Über das \gls{PhyPiDAQ} werden, die an das \gls{RasPi} angeschlossenen Sensoren adressiert. Die Sensoren schicken die \gls{messdaten} über das \gls{RasPi} an die Anwendung, die auf dem Computer läuft. 


\clearpage
\section{Produktumgebung}\label{produktumgebung}

Die Anwendung läuft auf einem Computer. Die \gls{messdaten} werden über \glspl{sensor} an einem \gls{RasPi} erfasst.

\subsection{Software}

Die Anwendung läuft auf Computern mit den Betriebssystemen Linux ab Kernel 4.15 oder Microsoft Windows ab Windows 10.
Ebenfalls muss eine \gls{JVM} Version 8 installiert sein. Die Anwendung muss auf dem Computer vollständig installiert sein. 

\subsection{Hardware}

Die Anwendung läuft auf Computern, welche für den generellen Einsatz in der Schule gedacht sind.
Die enthaltenen Komponenten sollten beispielsweise ein aktueller 4-Kern-Prozessor, 8 GB RAM und
mindestens 256 GB Speicherplatz sein.\footnote{Beispielkonfiguration eines Schulcomputers nach dem Beraterkreis zur IT-Ausstattung von Schulen des Bayrischen Staatsministerium für Bildung und Kultus, Wissenschaft und Kunst, Votum 2016 unter \url{https://www.mebis.bayern.de/infoportal/konzepte/it-ausstattung/votum/}} Die Mindestanforderungen an den Computer für die Anwendung sind jedoch geringer. Die dargestellten Komponenten stellen aber ein Computer dar, auf dem die Anwendung ohne Probleme läuft. Schnittstellen für USB und Netzwerk sind ebenfalls erforderlich. 
Auf dem über eine Schnittstelle verbundenen \gls{RasPi} des Models 3B+ muss \gls{PhyPiDAQ} installiert und verwendungsfähig sein.

\subsection{Schnittstelle}

\subsubsection{PhyPiDAQ}

Bei PhyPiDAQ handelt es sich um eine Anwendung zur Datenerfassung und Analyse mit einem \gls{RasPi}. Diese ist nicht Bestandteil des Produktes, wird jedoch zur Datenerfassung und Datenverarbeitung und somit zur Funktionalität der Anwendung  benötigt. Das Programm, welches in der Programmiersprache \gls{python3} geschrieben ist bietet einheitliche Schnittstellen zur Verwendung der Sensoren mit dem \gls{RasPi}.

\clearpage
\section{Funktionale Anforderungen}\label{funktionale}

Funktionale Anforderungen enthalten eine genaue und detaillierte Beschreibung der Produktfunktionen. Diese Produktfunktionen legen fest, was die Anwendung tun soll. Optionale funktionale Anforderungen sind Produktfunktionen, um welche die Anwendung erweitert werden könnte. Diese Produktfunktionen würden die Benutzerfreundlichkeit und Funktionalität verbessern.


\subsection{GUI} % Glossarverweise in Überschriften besser vermeiden

\begin{description}
\hypertarget{link-f010}{\item[F010]} Die Benutzer erreichen nach Öffnung der Anwendung direkt die \gls{GUI}, welche aus Menüleiste, Konfigurationsfeld und Darstellungsfeld besteht.
\hypertarget{link-f020}{\item[F020]} Die Anwendung ist über Optionen konfigurierbar. 
\hypertarget{link-f030}{\item[F030]} Die Anwendung besitzt eine Dateiverwaltung.
\hypertarget{link-f040}{\item[F040]} Die Anwendung bietet eine Hilfe-Oberfläche an.
\end{description}

\subsubsection{Menüfeld}

\begin{description}
\hypertarget{link-f050}{\item[F050]} Es gibt eine grafische Auswahl an vordefinierten \glspl{sensor}.
\hypertarget{link-f060}{\item[F060]} Es gibt eine grafische Auswahl an vordefinierten \glspl{transformation}.
\hypertarget{link-f070}{\item[F070]} Es gibt eine grafische Auswahl an vordefinierten \glspl{darstellung}.
\hypertarget{link-f080}{\item[F080] }Der Benutzer erhält, durch visuelle Repräsentationen der Konfigurationsbausteine, eine Darstellung ihrer Komplexität.
\end{description}

\subsubsection{Optional: Zusätzliche Funktionen im Menüfeld}

\begin{description}
\hypertarget{link-f090}{\item[(opt.) F090]} Der Benutzer soll in der graphischen Auswahl an Sensoren (\hyperlink{link-f050}{F050}) weitere Sensoren hinzufügen können.
\hypertarget{link-f100}{\item[(opt.) F100]} Der Benutzer soll in der graphischen Auswahl an Transformationen (\hyperlink{link-f060}{F060}) weitere Transformation hinzufügen können.
\hypertarget{link-f110}{\item[(opt.) F110]} Die Transformationen (\hyperlink{link-f100}{F100}) sollen in eigenen Python-Skripten geschrieben werden können.
\end{description}

\subsubsection{Konfigurationsfeld}

\begin{description}
\hypertarget{link-f120}{\item[F120]} Durch \gls{dragdrop} kann der Benutzer \glspl{Konfigurationsbaustein} im Konfigurationsfeld platzieren.
\hypertarget{link-f130}{\item[F130]} Der Benutzer kann eine Messung starten.
\end{description}


\subsubsection{Darstellungsfenster}

\begin{description}
\hypertarget{link-f140}{\item[F140]} Das Darstellungsfenster ist bei der Initialisierung der Anwendung leer.
\hypertarget{link-f150}{\item[F150]} Durch Starten einer Messung (\hyperlink{link-f130}{F130}) wird die visuelle Darstellung der Messung im Darstellungsfenster angezeigt.
\end{description}



\subsection{Konfigurationserstellung}

\begin{description}

\hypertarget{link-f160}{\item[F160]} Über die Dateiverwaltung (\hyperlink{link-f030}{F030}) kann der Benutzer eine gespeicherte \gls{Benutzerkonfiguration} öffnen oder alte Messwerte verwenden.
\hypertarget{link-f170}{\item[F170]} Durch Öffnen gespeicherter \gls{Benutzerkonfiguration} oder Messwerte (\hyperlink{link-f160}{F160}) werden geladene Konfigurationen und \gls{messdaten} werden automatisch nach Format überprüft.
\hypertarget{link-f180}{\item[F180]} Aus der graphischen Auswahl an Sensoren (\hyperlink{link-f050}{F050}) kann der Benutzer Sensoren durch \gls{dragdrop} in das Konfigurationsfeld ziehen.
\hypertarget{link-f190}{\item[F190]} Die Anwendung sollte automatisch überprüfen, ob der ausgewählte Sensor richtig angeschlossen ist.
\hypertarget{link-f200]}{\item[F200]} Durch die Überprüfung des Sensors (\hyperlink{link-f190}{F190}) wird dem Benutzer automatisch eine visuelle Rückmeldung gegeben.
\hypertarget{link-f210}{\item[F210]} Aus der graphischen Auswahl an Transformationen (\hyperlink{link-f060}{F060}) kann der Benutzer eine oder mehrere Transformationen durch \gls{dragdrop} in das Konfigurationsfeld ziehen.
\hypertarget{link-f220}{\item[F220]} Die ausgewählten Transformationen kann der Benutzer mit einem ausgewählten Sensor verknüpfen.
\hypertarget{link-f230}{\item[F230]} Aus der graphischen Auswahl an Transformationen (\hyperlink{link-f070}{F070}) kann der Benutzer eine Darstellungsart per \gls{dragdrop} in das Konfigurationsfeld zu ziehen.
\hypertarget{link-f240}{\item[F240]} Die Ausgewählte Darstellung kann der Benutzer mit einer Transformation verknüpfen.
\hypertarget{link-f250}{\item[F250]} Über die Dateiverwaltung (\hyperlink{link-f030}{F030}) kann der Benutzer seine eigene\gls{Benutzerkonfiguration} speichern.
\hypertarget{link-f260}{\item[F260]} Konfigurationsbausteine können durch \gls{dragdrop} wieder aus dem Konfigurationsfeld entfernt werden.

\end{description}

\subsubsection{Optional: Weiterentwicklung der Konfigurationserstellung}

\begin{description}

\hypertarget{link-f270}{\item[(opt.) F270]} Die Anwendung besitzt eine Check-Funktion, welche die \gls{Benutzerkonfiguration} vor einem Start auf Verwendbarkeit überprüft.
\hypertarget{link-f280}{\item[(opt.) F280]} Durch das Prüfen (\hyperlink{link-f270}{F270}) wird durch visuelle Rückmeldung dargestellt, ob die Messkonfiguration verwendet werden kann.

\end{description}

\subsection{Messablauf}

\begin{description}

\hypertarget{link-f290}{\item[F290]} Der Benutzer kann Messlänge und die Wertebereiche einer Messung festlegen.
\hypertarget{link-f300}{\item[F300]} Der Benutzer kann nur die \gls{Benutzerkonfiguration} starten, die sich aktuell im Konfigurationsfeld befindet.
\hypertarget{link-f310}{\item[F310]} Bei Starten der Messung (\hyperlink{link-f300}{F300}) wird automatisch überprüft ob die \gls{Benutzerkonfiguration} eine sinnvolle Kombination von Sensoren, Transformationen und Darstellungen ist.
\hypertarget{link-f320}{\item[F320]} Durch Starten der Messung (\hyperlink{link-f300}{F300}) wird automatisch mit der visuellen Darstellung der Messdaten begonnen.
\hypertarget{link-f330}{\item[F330]} Der Benutzer kann eine Messung löschen.
\hypertarget{link-f340}{\item[F340]} Bei Löschen einer Messung (\hyperlink{link-f330}{F330}) wird die Messung gestoppt und die visuelle Darstellung auf den Ausgangszustand gebracht.
\hypertarget{link-f350}{\item[F350]} Durch Löschen einer Messung (\hyperlink{link-f330}{F330}) wird nicht die \gls{Benutzerkonfiguration} gelöscht.
\hypertarget{link-f360}{\item[F360]} Durch Löschen einer Messung (\hyperlink{link-f330}{F330}) muss der Benutzer erst die Messung starten, um weiter zu messen.
\hypertarget{link-f370}{\item[F370]} Der Benutzer kann eine Messung pausieren.
\hypertarget{link-f380}{\item[F380]} Der Benutzer kann eine Messung fortsetzen.
\hypertarget{link-f390}{\item[F390]} Der Benutzer kann eine Messung nur fortsetzen (\hyperlink{link-f380}{F380}), wenn sie zuvor durch pausiert wurde (\hyperlink{link-f370}{F370}).
\hypertarget{link-f400}{\item[F400]} Der Benutzer kann die Messdaten speichern.
\hypertarget{link-f410}{\item[F410]} Der Benutzer kann den durch die Messung erzeugten Graphen speichern.
\hypertarget{link-f420}{\item[F420]} Bei Speichern der Messdaten (\hyperlink{link-f400}{F400}) und Speichern des Graphen [\hyperlink{link-f410}{F410}] öffnet sich automatisch das Verzeichnis und der Benutzer muss einen eigenen Dateinamen eingeben und speichern. 

\end{description}

\subsection{Fehlermeldungen}

\begin{description}
\hypertarget{link-f430}{\item[F430]} Bei Überprüfen des Formats (\hyperlink{link-f170}{F170}) wird bei einem falschen Format eine aussagekräftige Fehlermeldung zurückgegeben.
\hypertarget{link-f440}{\item[F440]} Die Anwendung sollte dem Benutzer automatisch beim Löschen einer Messung oder beim Schließen der Anwendung darauf hinweisen, dass die Messdaten ohne Speichern der Messdaten verloren gehen.
\hypertarget{link-f450}{\item[F450]} Die Anwendung sollte dem Benutzer eine aussagekräftige Fehlermeldung zurückgeben, falls es zu einem Datenabbruch der Messdaten kommt.
\hypertarget{link-f460}{\item[F460]} Bei Überprüfung der \gls{Benutzerkonfiguration} bei Start einer Messung (\hyperlink{link-f310}{F310}) wird bei einer \gls{Benutzerkonfiguration}, welche nicht verwendet werden kann, eine aussagekräftige Fehlermeldung zurückgegeben.

\end{description}

\subsection{Bedienungshilfen}

\begin{description}

\hypertarget{link-f470}{\item[F470]} Die Konfigurationsbausteine enthalten Informationen und Hilfestellung, die der Benutzer aufrufen kann.
\hypertarget{link-f480}{\item[F480]} Über die Hilfe-Oberfläche (\hyperlink{link-f040}{F040}) erhält der Nutzer eine kurze Beschreibung zur Funktionalität der Anwendung.

\end{description}

\subsection{Sprache}

\begin{description}

\hypertarget{link-f490}{\item[F490]} Die Anwendung ist in deutscher Sprache.

\end{description}

\subsubsection{Optional: Internationalisierung}

\begin{description}

\hypertarget{link-f500}{\item[(opt.) F500]} Die Anwendung stellt dem Benutzer weitere Sprachpakete für die \gls{GUI} zur Verfügung.
\hypertarget{link-f510}{\item[(opt.) F510]} Durch die Optionen (\hyperlink{link-f020}{F020}) ist die Sprache für den Benutzer änderbar.

\end{description}

\subsection{Sonstiges}


\begin{description}

\hypertarget{link-f520}{\item[F520]} Die in der Anwendung enthaltenen Farben sollten, mit Rücksicht auf Benutzer mit Rot-Grün Schwäche oder Farbenblindheit, ein im Bezug auf diese Aspekte, barrierefreies Verwendung der Anwendung ermöglichen.

\end{description}

\subsubsection{Optional: Sonstiges}

\begin{description}

\hypertarget{link-f530}{\item[(opt.) F530]} Durch die Optionen (\hyperlink{link-f020}{F020}) sollte der Benutzer, die in der Anwendung verwendeten Farben, ändern können.
\hypertarget{link-f540}{\item[(opt.) F540]} Durch die Optionen (\hyperlink{link-f020}{F020}) sollte der Benutzer, die in der Anwendung verwendete Buchstabengröße, verändern können.

\end{description}

\clearpage
\section{Produktdaten}\label{produktdaten}

Zu speichern sind ausschließlich:

\begin{description}

\item[D010] \glspl{Benutzerkonfiguration}
\item[D020] \gls{messdaten} und deren Darstellungen
\item[D030] Benutzererstellte Bausteinprototypen (beinhaltet beispielsweise neu angepasste Sensorkonfigurationsdaten im Falle eines Protoypen vom Typ Sensor)
\item[D040] Lokaler Datenfluss bei einem Baustein während eines Messlaufs 

\end{description}

\clearpage
\section{Nichtfunktionale Anforderungen}\label{nichtfunktionale}

Nichtfunktionale Anforderungen gehen über funktionale Anforderungen hinaus, indem sie Aspekte beschreiben, die sich nicht direkt auf das funktionale Verhalten des Systems beziehen. Unter anderem zählen dazu gewisse Qualitätsmerkmale und Zusicherungen für die Leistung des Produkts.

\subsection{Qualitätsanforderungen}\label{quali}

Auf folgende Qualitätsmerkmale wird entsprechend besonders Wert gelegt:

\begin{tabular}{ | c | c | c | c | c | }
	\hline
	\textbf{Merkmal} & \textbf{sehr wichtig} & \textbf{wichtig} & \textbf{weniger wichtig} & \textbf{unwichtig} \\
	\hline
	Robustheit & X & & & \\
	Zuverlässigkeit & & X & & \\
	Benutzerfreundlichkeit & X & & & \\
	Sicherheit & & & & X \\
	Erweiterbarkeit & & X & & \\
	Effizienz & & X & & \\
	Portierbarkeit & & & X & \\
	Kompatibilität & & & X & \\
	\hline
\end{tabular}

\begin{tabular}{| p{3cm} | p{10cm} |}
	\hline
	\textbf{Merkmal} & \textbf{Begründung für die Werteverteilung}\\
	\hline
	Robustheit & Durch eine robuste Anwendung sollen möglichst wenige Komplikationen während des Unterrichtes auftreten. Der Lehrer soll sich nicht mit Softwareproblemen beschäftigen müssen. Das Qualitätsmerkmal ist deshalb sehr wichtig, da von unvorhergesehenen Situationen, wie etwa Fehlbedienung, ausgegangen werden muss.\\\cline{1-2}
	Zuverlässigkeit & Durch die Zuverlässigkeit der Anwendung soll der Betrieb in der vorgesehenen Umgebung (siehe Kapitel \ref{produktumgebung}) gewährleistet werden. Das Qualitätsmerkmal ist deshalb wichtig, da eine Anwendung, die in einer erwarteten Umgebung bereits scheitert, nicht robust sein kann.\\\cline{1-2}
	Benutzer-\newline freundlichkeit & Da die Anwendung einen wissensvermittelnden Auftrag hat, ist es sehr wichtig, dass die Anwendung für Schüler verständlich ist. Die Schüler sollen durch eine hohe Benutzerfreundlichkeit dazu angehalten werden, sich selbstständig mit den wissenschaftlichen Themen zu beschäftigen. Umgekehrt könnte das Fehlen eines hohen Grads an Benutzerfreundlichkeit ein Scheitern der Anwendung zur Folge haben.\\\cline{1-2}
	Sicherheit & Die Anwendung soll ausschließlich in einer sicheren Umgebung verwendet werden. Das bereits vorhandene Berechtigungssystem des Betriebssystems sollte daher ausreichen. Das Qualitätsmerkmal ist somit unwichtig.\\\cline{1-2}
	Erweiterbarkeit & Es ist angedacht, die Anwendung um weitere Elemente, wie etwa Sensoren, erweitern zu können. Außerdem sollen Schüler eigene Transformationen erstellen und diese auch speichern können. Es ist daher wichtig, dass die Anwendung mit diesen Aufgaben umgehen kann.\\\cline{1-2}
	Effizienz & Im Vergleich zu einem PC bietet ein Raspberry Pi lediglich beschränkte Ressourcen. Effizienz bedeutet in diesem Kontext, dass erstellte und importierte Software-Artefakte das Benutzererlebnis trotz knapper Ressourcen maximieren. Das Qualitätsmerkmal ist somit wichtig.\\\cline{1-2}
	Portierbarkeit & Zum aktuellen Zeitpunkt ist keine Portierung der Anwendung auf viele unterschiedliche Betriebssysteme vorgesehen. Das Qualitätsmerkmal ist deshalb weniger wichtig.\\\cline{1-2}
	Kompatibilität & Da die Anwendung die Daten von einer externen Quelle, wie zum Beispiel \gls{PhyPiDAQ} bekommt, muss sie nicht stark auf sich verändernde Teile reagieren. Das Qualitätsmerkmal ist daher weniger wichtig.\\\cline{1-2}
	\hline
\end{tabular}

\subsection{Produktleistungen}

\begin{description}

\item[NF010] Auslesen von \gls{messdaten} von einem einzelnen Sensor innerhalb von 50 ms.
\item[NF020] Alle 20 ms können neue Messdaten angefordert werden.
\item[NF030] Bis zu drei \glspl{sensor} können zeitgleich verwendet werden, ohne die Leistung zu beeinträchtigen.
\item[NF040] Mehrere \glspl{transformation} können sukzessive voneinander abhängig sein. Zyklische Abhängigkeiten sind nicht erlaubt.
\item[NF045] Verarbeitung von Daten benötigt maximal 10 ms pro Transformation.
\item[NF050] Mehrere \glspl{darstellung} können zeitgleich verwendet werden.
\item[NF055] Darstellung von Daten benötigt maximal 30 ms pro Darstellungsbaustein.
\item[NF060] Die Anwendung reagiert auf Benutzereingaben innerhalb von zwei Sekunden; mindestens mit einem Zwischenergebnis, Ladebalken oder ähnlichem.

\end{description}

\subsection{Benutzbarkeit}

\begin{description}

\item[NF070] Die \gls{GUI} ist ergonomisch gestaltet, orientiert an ISO 9241.
\item[NF080] Nichttriviale Funktionen werden dem Benutzer erklärt.
\item[NF090] Ungespeicherte Daten werden nicht ohne Warnung verworfen.
\item[NF100] Über Programmfehler wird der Benutzer durch aussagekräftige Fehlermeldungen informiert.
\item[NF110] Schriftgröße von Text kann verändert werden.
\item[NF120] Farbschema von farbigen GUI-Elementen ist veränderbar.
\item[NF130] Die Sprache der Benutzeroberfläche kann verändert werden.

\end{description}

\subsection{Zuverlässigkeit und Robustheit}

\begin{description}

\item[NF140] Konsistenz: Bei gleichen Vorbedingungen führen identische Eingaben auch zum gleichen Ergebnis.
\item[NF150] Korrektheit: Bei korrekten Eingaben werden die erwarteten funktionalen Anforderungen erfüllt.
\item[NF160] Unerwartete und fehlerhafte Eingaben führen nicht zum Absturz der Anwendung.
\item[NF170] Verbindungsverlust von \glspl{sensor} führt nicht zum Absturz der Anwendung.
\item[NF180] Unerwartete \gls{messdaten} (z.B. außerhalb eines eingestellten Wertebereichs) führen nicht zum Absturz der Anwendung.

\end{description}

\subsection{Sonstige}

\begin{description}

\item[NF190] Die Software erfüllt die Definition von \gls{opensource}- und freier Software.\footnote{\url{https://opensource.org/osd} und \url{https://www.gnu.org/philosophy/free-sw.de.html}}
\item[NF200] Die Software ist erweiterbar und anpassbar an veränderte Umstände, auch durch Außenstehende.
\item[NF210] Die \gls{DSGVO} wird nicht verletzt; d.h. personenbezogene Daten werden ausschließlich dann verarbeitet, wenn sie vom Benutzer freiwillig eingespeist werden.
\item[NF220] Daten werden ausschließlich lokal gespeichert.

\end{description}

\clearpage
\section{Globale Testfälle und Testszenarien}\label{testfall}

\subsection{Einführung}

In diesem Kapitel werden einige Testfälle zur Benutzung der Anwendung und zur Handhabung von Fehlern dargestellt. Zum besseren Verständnis werden noch einmal einige verwendete Begriffe und der generelle Ablauf einer Nutzung erläutert. Die Anwendung bezeichnet die gesamte in diesem Projekt erstellte Software. Sie benutzt \gls{Benutzerkonfiguration}en um einen Versuchsaufbau zu modellieren. Eine Messkonfiguration besteht aus Bausteinen(\gls{sensor}, \gls{transformation}, \gls{darstellung}) und deren Verbindungen untereinander. Die Messkonfigurationen werden im Konfigurationsfeld erstellt. Außerdem können sie geladen und gespeichert werden. Beim Starten überprüft die Anwendung, ob eine gültige Messkonfiguration vorliegt. Um die resultierenden Daten und Graphen zu speichern, muss die Messung gestoppt werden.

\subsection{Testfälle zur Nutzung der Anwendung}

\begin{description} 

\hypertarget{link-t010}{\item[T010]}Starten der Anwendung und Hilfe

\begin{figure}[htbp]
	\begin{center}
		\includegraphics[width = 10cm]{Grafik/T010-Ablauf.png}
		\caption{Kurze Darstellung des Ablaufs von Testfall \textbf{T010}.}
		\label{T010-Ablauf}
	\end{center}
\end{figure}
\begin{itemize}

\item []\textbf{Testziel:} Teste das Verhalten der Anwendung aus der Sicht eines Benutzers, der diese zum ersten Mal verwendet. Der Ablauf ist in Abbildung \ref{T010-Ablauf} zu sehen.

\item []\textbf{Vorbedingung:} Die Anwendung ist installiert. Zu Sehen ist der Desktop des Benutzers mit einer Verknüpfung zur Anwendung.
\item []\textbf{Aktion:} Der Benutzer öffnet die Anwendung über die Verknüpfung. Danach informiert er sich über die Anwendung mittels der Hilfe.
\item []\textbf{Reaktion:} Die Anwendung wird geöffnet. Dabei überprüft diese, ob ein \gls{RasPi} angeschlossen ist und stellt das Hauptfenster mit leerem Konfigurationsfeld und leeren Darstellungen dar. Jeder dargestellte Text ist auf Deutsch. Nach dem Öffnen der Hilfe, werden dem Benutzer die nötigen Informationen zur Bedienung der Anwendung angezeigt.
  
\item []\textbf{Nachbedingung:} Die Anwendung ist geöffnet. Zu Sehen ist das offene Hilfefenster. 

\item []\textbf{Ergebnis:} Der Benutzer kann die Anwendung starten. Die Bedienung der Anwendung ist komplett auf Deutsch möglich.
\item []\textbf{Wichtige abgedeckte Funktionale Anforderungen} \hyperlink{link-f010}{\textbf{F010}} erreiche GUI nach Start, \hyperlink{link-f140}{\textbf{F140}} leere Darstellung nach Anwendungsstart, \hyperlink{link-f480}{\textbf{F480}} Hilfe zu Anwendung, \hyperlink{link-f490}{\textbf{F490}} Texte der Anwendung auf Deutsch

\end{itemize}


\hypertarget{link-t020}{\item[T020]} Starten der Demo 

\begin{figure}[htbp]
	\begin{center}
		\includegraphics[width = 10cm]{Grafik/T020-Ablauf.png}
		\caption{Kurze Darstellung des Ablaufs von Testfall \textbf{T020}.}
		\label{T020-Ablauf}
	\end{center}
\end{figure}
\begin{itemize}

\item []\textbf{Testziel:} Teste das Verhalten der Anwendung beim Ausführen der Demo. Die Demo nutzt Daten aus einer Datei und benötigt keinen angeschlossenen \gls{RasPi}. Der Ablauf ist in Abbildung \ref{T020-Ablauf} zu sehen.

\item []\textbf{Vorbedingung:} Die Anwendung ist gestartet worden. Zu Sehen ist das offene Hauptfenster mit leerem Konfigurationsfeld.
\item []\textbf{Aktionen:} Der Benutzer öffnet die Messkonfigurationsdatei der Demo mithilfe der Ladefunktion der Systemleiste. Sobald die Messkonfiguration geladen ist, startet der Benutzer die Messung.
\item []\textbf{Reaktionen:} Nach dem Laden wird die Messkonfiguration in dem Konfigurationsfeld angezeigt. Nach dem Starten der Messung, werden die verarbeiteten Daten im Graph dargestellt. Dabei wird auch überprüft, ob die Messkonfiguration gültig ist.
\item []\textbf{Nachbedingung:} Die Anwendung ist offen. Eine gültige Messkonfiguration ist in der Konfigurationsfeld geladen. Die Messung ist im laufenden Zustand und die Ergebnisse werden richtig dargestellt.

\item []\textbf{Ergebnis:} Die Anwendung kann eine Messkonfiguration, die nur Daten aus einer Datei benötigt, problemlos ohne angeschlossenem \gls{RasPi} durchführen. 
\item []\textbf{Wichtige abgedeckte Funktionale Anforderungen, die in den vorangegangenen Testfällen noch nicht abgedeckt wurden:} \hyperlink{link-f130}{\textbf{F130}} Starten einer Messung, \hyperlink{link-f160}{\textbf{F160}} Laden einer Messkonfiguration

\end{itemize}



\hypertarget{link-t030}{\item[T030]} Lehrer erstellt und speichert eine Messkonfiguration

\begin{figure}[htbp]
	\begin{center}
		\includegraphics[width = 13cm]{Grafik/T030-Ablauf.png}
		\caption{Kurze Darstellung des Ablaufs von Testfall \textbf{T030}.}
		\label{T030-Ablauf}
	\end{center}
\end{figure}




\begin{itemize}

\item []\textbf{Testziel:} Teste eine typische Verwendung der Software anhand des gegebenen Szenarios. Dabei wird das Erstellen, Verändern und Speichern einer Messkonfiguration getestet. Der Ablauf ist in Abbildung \ref{T030-Ablauf} zu sehen.

\item []\textbf{Vorbedingung:} Die Anwendung ist gestartet und es liegt keine Messkonfiguration vor. Die Anwendung hat Zugriff auf einen laufenden \gls{RasPi} mit \gls{PhyPiDAQ}. Die Sensoren die verwendet werden sollen sind ordnungsgemäß angeschlossen.
\item []\textbf{Aktionen:} Der Lehrer zieht zwei angeschlossene Sensoren und eine Transformation in das Konfigurationsfeld. Danach ändert er einige Einstellungen an der Messkonfiguration. Das Ergebnis speichert er über die Systemleiste als Messkonfigurationsdatei.
\item []\textbf{Reaktion:} Nach jedem Hinzufügen eines Sensors oder einer Transformation, wird an der entsprechenden Stelle der jeweilige Konfigurationsbaustein sichtbar. Dabei prüft die Anwendung, ob ein Sensor angeschlossen ist. Nach dem Speichern der Messkonfiguration sieht der Lehrer die erstellte Datei. Dabei prüft die Anwendung ob die angepassten Einstellungen gültig sind. Die Messkonfigurationsdatei wird erstellt, unabhängig ob die Messkonfiguration vollständig oder gültig ist.
\item []\textbf{Nachbedingung:}  Die Anwendung ist geöffnet. Im Hauptfenster wird die erstelle Messkonfiguration angezeigt. Diese ist auch als Datei im entsprechenden Pfad gespeichert.

\item []\textbf{Ergebnis:} Die Anwendung kann eine Messkonfiguration als Datei speichern und prüfen, ob ein Sensor angeschlossen ist.
\item []\textbf{Wichtige abgedeckte funktionale Anforderungen, die in den vorangegangenen Testfällen noch nicht abgedeckt wurden:} \hyperlink{link-f180}{\textbf{F180}} füge Sensor hinzu, \hyperlink{link-f190}{\textbf{F190}} prüfe ob Sensor angeschlossen, \hyperlink{link-f210}{\textbf{F210}} füge Transformation hinzu, \hyperlink{link-f250}{\textbf{F250}} speichere Messkonfiguration, \hyperlink{link-f290}{\textbf{F290}} Einstellungen Messkonfiguration 

\end{itemize}

\hypertarget{link-t040}{\item[T040]} Schüler bearbeitet Aufgabe

\begin{figure}[htbp]
	\begin{center}
		\includegraphics[width = 13cm]{Grafik/T040-Ablauf.png}
		\caption{Kurze Darstellung des Ablaufs von Testfall \textbf{T040}.}
		\label{T040-Ablauf}
	\end{center}
\end{figure}
\begin{itemize}

\item []\textbf{Testziel:} Teste die Anwendung aus Sicht eines Schülers, der eine Aufgabe bearbeiten soll.

\item []\textbf{Vorbedingung:} Die Anwendung ist gestartet und es liegt keine Messkonfiguration vor. Die Anwendung hat Zugriff auf einen laufenden \gls{RasPi} mit \gls{PhyPiDAQ}. Die Sensoren die verwendet werden sollen sind ordnungsgemäß angeschlossen. Der Ablauf ist in Abbildung \ref{T040-Ablauf} zu sehen.

\item []\textbf{Aktionen:} Der Schüler lädt über die Systemleiste die Messkonfiguration der Aufgabe. Danach informiert er sich über die Sensoren, Transformationen und Darstellungen über die angebotene Hilfestellungen. Als Nächstens vollendet er die Messkonfiguration.
 
\item []\textbf{Reaktionen:} Nach dem Laden sieht der Schüler die Messkonfiguration in dem Konfigurationsfeld. Dabei prüft die Anwendung beim Laden der Messkonfiguration, ob die Sensoren angeschlossen sind.  Nach dem Aufrufen der Hilfe, wird die entsprechende Hilfestellung angezeigt. Wird die Messkonfiguration durch den Schüler verändert, aktualisiert die Anwendung die grafische Darstellung der Messkonfiguration entsprechend. 
\item []\textbf{Nachbedingung:} Die Anwendung ist geöffnet. Es wird das Hauptfenster angezeigt. Dabei wird die vom Schüler erstellte Messkonfiguration im Konfigurationsfeld angezeigt.


\item []\textbf{Ergebnis:} Der Benutzer kann Messkonfigurationen laden, diese verändern und durch das Erstellen auf Gültigkeit prüfen. 
\item []\textbf{Wichtige abgedeckte funktionale Anforderungen, die in den vorangegangenen Testfällen noch nicht abgedeckt wurden:} \hyperlink{link-f230}{\textbf{F230}} füge Darstellung hinzu, \hyperlink{link-f470}{\textbf{F470}} Hilfe zu Bausteinen

\end{itemize}

\hypertarget{link-t050}{\item[T050]} Schüler startet Messung und speichert Ergebnisse

\begin{figure}[htbp]
	\begin{center}
		\includegraphics[width = 13cm]{Grafik/T050-Ablauf.png}
		\caption{Kurze Darstellung des Ablaufs von Testfall \textbf{T050}.}
		\label{T050-Ablauf}
	\end{center}
\end{figure}
\begin{itemize}

\item []\textbf{Testziel:} Teste die Anwendung aus Sicht eines Schülers, der Messung starten und deren Ergebnisse speichern soll. Der Ablauf ist in Abbildung \ref{T050-Ablauf} zu sehen.

\item []\textbf{Vorbedingung:} Die Anwendung ist gestartet und es ist die in \hyperlink{link-t040}{\textbf{T040}} erstelle Messkonfiguration geladen worden.

\item []\textbf{Aktionen:} Der Schüler startet die Messung. Während der Messung beeinflusst er die Daten durch einen Sensor. Am Ende der Messung speichert der Schüler den resultierenden Graphen und die Werte als Dateien ab.
\item []\textbf{Reaktionen:} Nach dem Starten, stellt die Anwendung die verarbeiteten Daten als Graphen dar. Dabei prüft die Anwendung vor dem Starten der Messung, ob die Messkonfiguration gültig ist. Veränderungen der Messdaten werden, nach der Verarbeitung, im aktualisierten Graph dargestellt. Die Anwendung erstellt nach dem Speichern die Dateien, die die Ergebnisse enthalten.
\item []\textbf{Nachbedingung:}  Die Anwendung ist geöffnet. Es wird das Hauptfenster angezeigt. Im Konfigurationsfeld ist die Messkonfiguration geladen. Die Messung ist gestoppt. Die gespeicherten Ergebnisse sind als Datei vorhanden.


\item []\textbf{Ergebnis:} Der Benutzer die Messung einer gültigen Messkonfiguration starten. Die Ergebnisse kann er als Graphen oder als Werte speichern. 
\item []\textbf{Wichtige abgedeckte funktionale Anforderungen, die in den vorangegangenen Testfällen noch nicht abgedeckt wurden:} \hyperlink{link-f300}{\textbf{F300}} Starte Messung, \hyperlink{link-f320}{\textbf{F320}} stelle Ergebnisse dar, \hyperlink{link-f400}{\textbf{F400}} speichere Messdaten, \hyperlink{link-f410}{\textbf{F410}} speichere Messgraph


\end{itemize}
\end{description}

\subsection{Testfälle zur Handhabung von Fehlern}

\begin{description} 

\hypertarget{link-t200}{\item[T200]} Laden einer ungültigen Datei als Messkonfiguration

\begin{figure}[htbp]
	\begin{center}
		\includegraphics[width = 7cm]{Grafik/T200-Ablauf.png}
		\caption{Kurze Darstellung des Ablaufs von Testfall \textbf{T200}.}
		\label{T200-Ablauf}
	\end{center}
\end{figure}
\begin{itemize}

\item []\textbf{Testziel:} Teste das Verhalten der Anwendung beim Laden von ungültigen Messkonfigurationsdateien. Der Ablauf ist in Abbildung \ref{T200-Ablauf} zu sehen.

\item []\textbf{Vorbedingung:} Geöffnete Anwendung.
\item []\textbf{Aktionen:} Der Benutzer öffnet den Dialog zum Laden einer Datei über die Systemleiste und wählt eine Datei mit ungültigem Format zum Laden aus.
\item []\textbf{Reaktionen:} Die Anwendung überprüft, ob die Datei ein gültiges Format hat und verhindert das Laden von ungültigen Formaten. Dabei wird  im Fehlerfall eine aussagekräftige Fehlermeldung ausgegeben.
\item []\textbf{Nachbedingung:} Die Anwendung ist offen. Es wird das Hauptfenster angezeigt. Dabei ist das Konfigurationsfeld leer und es wird die entsprechende Fehlermeldung angezeigt. 


\item []\textbf{Ergebnis:} Die Anwendung kann beim Laden von Messkonfigurationen zwischen gültigen und ungültigen Dateiformaten unterscheiden und entsprechend reagieren.
\item []\textbf{Wichtige abgedeckte funktionale Anforderungen, die in den vorangegangenen Testfällen noch nicht abgedeckt wurden:} \hyperlink{link-f430}{\textbf{F430}} überprüfe Format bei Laden

\end{itemize}

\hypertarget{link-t210}{\item[T210]} Starten einer ungültigen Messkonfiguration

\begin{figure}[htbp]
	\begin{center}
		\includegraphics[width = 10cm]{Grafik/T210-Ablauf.png}
		\caption{Kurze Darstellung des Ablaufs von Testfall \textbf{T210}.}
		\label{T210-Ablauf}
	\end{center}
\end{figure}
\begin{itemize}

\item []\textbf{Testziel:} Teste das Verhalten der Anwendung beim Starten einer ungültigen Messkonfiguration. Der Ablauf ist in Abbildung \ref{T210-Ablauf} zu sehen.

\item []\textbf{Vorbedingung:} Das Hauptfenster der Anwendung ist geöffnet. Die verwendeten Sensoren sind angeschlossen. 

\item []\textbf{Aktionen:} Der Benutzer konstruiert eine Messkonfiguration aus zwei Sensoren, einer Transformation und einer Darstellung. Allerdings verbindet er die Darstellung nicht mit dem Rest der Konstruktion und startet die Messkonfiguration.
\item []\textbf{Reaktionen:} Die Anwendung überprüft vor dem Starten der Messkonfiguration, ob diese gültig ist. Da diese ungültig ist, wird  die Messung nicht gestartet, sondern eine aussagekräftige Fehlermeldung wird ausgegeben.
\item []\textbf{Nachbedingung:} Die Anwendung ist offen. Es wird das Hauptfenster angezeigt. Dabei ist im Konfigurationsfeld die Messkonfiguration geladen. Die Messung ist nicht am Laufen und es wird die entsprechende Fehlermeldung angezeigt. 

\item []\textbf{Ergebnis:} Die Anwendung kann beim Starten zwischen gültigen und ungültigen Messkonfigurationen unterscheiden und entsprechend reagieren.
\item []\textbf{Wichtige abgedeckte funktionale Anforderungen, die in den vorangegangenen Testfällen noch nicht abgedeckt wurden:} \hyperlink{link-f310}{\textbf{F310}} überprüfe Messkonfiguration beim Start

\end{itemize}

\hypertarget{link-t220}{\item[T220]} Entfernen eines Sensors bei laufender Messung

\begin{figure}[htbp]
	\begin{center}
		\includegraphics[width = 10cm]{Grafik/T220-Ablauf.png}
		\caption{Kurze Darstellung des Ablaufs von Testfall \textbf{T220}.}
		\label{T220-Ablauf}
	\end{center}
\end{figure}
\begin{itemize}

\item []\textbf{Testziel:} Teste das Verhalten der Anwendung, wenn ein Sensor ausfällt und dessen Datenstrom abbricht. Der Ablauf ist in Abbildung \ref{T220-Ablauf} zu sehen.

\item []\textbf{Vorbedingung:} Die Anwendung ist geöffnet. Alle verwendeten Sensoren sind angeschlossen und betriebsbereit. Eine gültige Messkonfiguration aus zwei Sensoren, einer Transformation und einer Darstellung wurde geladen.
\item []\textbf{Aktionen:} Der Benutzer startet die Messung. Die Verbindung zu einem Sensor wird getrennt.
\item []\textbf{Reaktionen:} Die Anwendung erkennt, dass ein Sensor keine Daten mehr sendet. Die Messung stoppt. Eine aussagekräftige Fehlermeldung wird ausgegeben.
\item []\textbf{Nachbedingung:} Die Anwendung ist offen. Es wird das Hauptfenster angezeigt. Dabei ist im Konfigurationsfeld die Messkonfiguration geladen. Die Messung ist nicht am Laufen und es wird die entsprechende Fehlermeldung angezeigt. Ein Sensor ist als nicht verbunden markiert.

\item []\textbf{Ergebnis:} Die Anwendung kann mit dem ungewollten Verlust eines Datenstroms umgehen ohne abzustürzen.
\item []\textbf{Wichtige abgedeckte funktionale Anforderungen, die in den vorangegangenen Testfällen noch nicht abgedeckt wurden:} \hyperlink{link-f450}{\textbf{F450}} Melde Abbruch des Datenstrom

\end{itemize}


\end{description}

\clearpage
\section{Szenario mit Use-Case-Diagramm}\label{scenario}

Abbildung \ref{UseCaseDiagramm} zeigt die Use-Cases und deren Ablauf für das nun im Folgenden beschriebene Szenario. Akteure in diesem Szenario sind ein Lehrer, ein Schüler sowie ein weiterer Lehrer. Das Szenario verwendet weitestgehend Testfälle aus Kapitel \ref{testfall}. (Bausteinprototypen sind lediglich in Sollkriterien spezifiziert und werden von den Testfällen in Kapitel \ref{testfall} nicht abgedeckt.)

\begin{enumerate}
	\item Lehrer A kennt das Messkonfigurationswerkzeug noch nicht so gut und erstellt zunächst eine ungültige \gls{Benutzerkonfiguration}. 
	
	\item Der zweite Versuch gelingt und Lehrer A erstellt erfolgreiche eine Messkonfiguration, die sich gut als Basis für seinen Unterricht eignet. Diese Messkonfiguration speichert er sogleich auf seinem Speicherstick.
	
	\item Am nächsten Tag gibt Lehrer A den Speicherstick einem seiner Schüler. 
	
	\item Dieser soll die vorbereitete Aufgabe bearbeiten. Also öffnet er die Messkonfiguration und bearbeitet diese.
	
	\item Da Lehrer A zunächst Zweifel hat, dass sein Schüler die Aufgabe ordnungsgemäß gelöst hat, lässt er ihn einen Messlauf durchführen. Der Messlauf verläuft wie erwartet, und weil der Schüler sehr ordentlich ist, speichert er die Messergebnisse.
	
	\item Es hat sich herumgesprochen, dass die Schüler von Lehrer A ganz begeistert von seinem Physik-Unterricht sind, und deshalb möchte Lehrer B einen ähnlichen Unterricht machen. Da er nicht so wie Lehrer A bei Null anfangen möchte, lässt er sich die eingangs von Lehrer A erstellte Messkonfiguration auf einem Speicherstick geben. 
	
	\item Lehrer B öffnet die Messkonfiguration und schaut sie sich genau an.
	
	\item Die Messkonfiguration enthält zwar einen brauchbaren Darstellungsbaustein. Da Lehrer B aber ein paar sehr schwache Schüler in seiner Klasse hat, möchte ein paar Aspekte auf den Darstellungen, die er erhält, besser hervorheben. Und er sieht schon kommen, dass er (und evtl. auch seine Kollegen) in Zukunft noch öfter von der verbesserten Darstellungsart profitieren könnten. Also erstellt er einen neuen optimierten Bausteinprototyp für Darstellungen.
	
	\item Lehrer B ersetzt den alten Darstellungsbaustein durch einen mit der verbesserten Darstellungsart. 
	
	\item Da die neue Darstellungsart auch in Zukunft noch öfter zum Einsatz kommen soll, speichert Lehrer B sie persistent. 
	
	\item Im darauffolgenden Schuljahr bekommt Lehrer A wieder eine Klasse der gleichen Klassenstufe zugewiesen, und dieses Mal hat auch er das Pech, ein paar sehr schwache Schüler in seiner Klasse zu haben. Er erzählt Lehrer B davon, und dieser überlässt ihm per Speicherstick eine Kopie der Datei, welche den bereits vor einem Jahr erstellten Bausteinprototypen enthält. 	
	
	\item Lehrer A öffnet die Datei und erkennt, dass die neuen Darstellungsbausteine, die er seinen Messkonfigurationen nun hinzufügen kann, sehr viel anschaulicher sind.
\end{enumerate}


\begin{figure}[h]
	\begin{center}
		\includegraphics[width = 12cm]{Grafik/UseCaseDiagram.png}
		\caption{Use-Case-Diagramm}
		\label{UseCaseDiagramm}
	\end{center}
\end{figure}

\clearpage
\section{Systemmodelle}\label{systemmodell}

Das projektierte Gesamtsystem wird aller Voraussicht nach auf drei Systemprozesse verteilt sein, die wie in Abbildung \ref{DeploymentDiagram} dargestellt, miteinander Daten austauschen.

Einer dieser drei Systemprozesse enthält die \gls{JVM}, auf der die zu erstellende Hauptkomponente läuft. Diese enthält wiederum die grafische \gls{GrafBenOber}, mit deren Hilfe Benutzer Messkonfigurationen entwerfen und den Betrieb von Messläufen steuern und überwachen können.

Zum Ansteuern und Auslesen der Sensoren auf dem \gls{RasPi} wird das auf Python basierende Framework \gls{PhyPiDAQ} bereitgestellt. Dieses unterstützt bereits eine  ganze Reihe von Sensoren zur Messung diverser physikalischer Größen der Mechanik, Thermodynamik und Elektrodynamik. Da der Python-Interpreter im Allgemeinen nicht im selben Systemprozess wie die \gls{JVM} ausgeführt werden kann, wird für PhyPiDAQ ein zweiter Systemprozess benötigt.

Um Ressourcenknappheit auf dem \gls{RasPi} aus dem Weg zu gehen, aber auch zur Vereinfachung von Entwicklung und Demonstration, soll die Hauptkomponente nicht zwingend so wie PhyPiDAQ auf dem \gls{RasPi} ausgeführt werden müssen. 

Falls die Hauptkomponente tatsächlich auf einem anderen Rechner ausgeführt werden sollte, so muss sie mit dem Python-Interpreter hardware- bzw. plattformübergreifend Daten austauschen können. Bisher stellt PhyPiDAQ hierfür jedoch keine Funktionalität zur Verfügung. 

Als Lösungsalternativen bieten sich hier die Kommunikation über eine Netzwerkschnittstelle wie z. B. WLAN und die Kommunikation über eine Peripherieschnittstelle wie z. B. USB an.

Es wäre sicher möglich, PhyPiDAQ um den benötigten Programmcode zur anwendungsspezifischen Kommunikation zu erweitern. (In diesem Fall hätte das projektierte Gesamtsystem lediglich zwei Systemprozesse.) Lukrativer erscheint es aber, eine Utility-Softwarekomponente zur Datenvermittlung zu implementieren, die in einem dritten Systemprozess auf demselben Rechner wie PhyPiDAQ, also auf dem Raspberry Pi läuft. Sofern diese einfach gehalten und robust implementiert ist, kann diese zusätzlich eine Überwachungsfunktion bezüglich PhyPiDAQ übernehmen und dieses im Falle von Abstürzen und Nicht-Responsivität erneut ausführen. In Abbildung \ref{DeploymentDiagram} und im Folgenden wird diese Utility-Komponente ,,Measurement-Server`` genannt.
\newpage
Es gibt also zwei Szenarien, die 3 Systemprozesse zu verteilen: 

\begin{itemize}
	\item Der Systemprozess mit der Hauptkomponente läuft einem PC, und die beiden Systemprozesse mit Measurement-Server und PhyPiDAQ laufen auf dem Raspberry Pi. 
	\item Alle drei Systemprozesse laufen auf dem Raspberry Pi.
\end{itemize}

Abbildung \ref{DeploymentDiagram} stellt die Verteilung der Systemkomponenten auf zwei Rechnern, also das erste Szenario dar.

Da sich PhyPiDAQ und Measurement-Server in jedem Fall auf derselben Plattform befinden, kann die Kommunikation zwischen den beiden Instanzen invariant unter Zuhilfenahme einer von Linux gegebenen  Inter-Prozess-Kommunikation erfolgen, beispielsweise mittels einer Verknüpfung von Standard-Eingabe und Standard-Ausgabe über eine ,,Pipe``.

Measurement-Server und Hauptkomponente befinden sich im Kontrast hierzu jedoch nicht zwingend auf derselben Plattform. Es besteht das Risiko, dass sich nicht wenige Code-Fragmente für den Datenaustausch bei den beiden Szenarien stark unterscheiden, beispielsweise in Bezug auf die zeitliche Koordination im anwendungsspezifischen Protokollablauf. Softwaretechnisch ist es deshalb das beste Vorgehen, eine leichte Austauschbarkeit der Implementierungsvarianten sicherzustellen, indem man diese hinter einer invarianten Schnittstelle verbirgt, so wie im Klassendiagramm \ref{ClassDiagram} dargestellt.

Ob es sich beispielsweise bei der verteilten Kommunikation um WLAN per Netzwerkadapter und bei der \gls{Stand-Alone-Kommunikation} um WLAN per \gls{Local-Loop} handelt, oder ob es sich (ebenso beispielsweise) bei der verteilten Kommunikation um USB und bei der Stand-Alone-Kommunikation um IPC handelt, sollte spätestens zu Beginn der Implementierungsphase entschieden werden.  


\begin{figure}[h]
	\begin{center}
		\includegraphics[width = 12cm]{Grafik/DeploymentDiagram.png}
		\caption{UML-Deployment-Diagramm}
		\label{DeploymentDiagram}
	\end{center}
\end{figure}

\begin{figure}[h]
	\begin{center}
		\includegraphics[width = 12cm]{Grafik/ClassDiagramForCommunication.png}
		\caption{UML-Klassen-Diagramm}
		\label{ClassDiagram}
	\end{center}
\end{figure}

\clearpage
\section{Benutzungsoberfläche}\label{gui}

\subsection{Ziel der Benutzeroberfläche}

Das Ziel der Benutzeroberfläche ist es dem Anwender eine intuitive Benutzung des Programmes zu ermöglichen. Da das Programm im Schulbetrieb eingesetzt werden soll, ist eine gute Verständlichkeit wichtig. So soll auch eine lange Einarbeitungszeit für den Anwender vermieden werden. Dabei soll keine Funktionalität verloren gehen.

\subsection{Allgemeines}

Um die Ziele zu erfüllen, wird das Programm über eine \gls{GrafBenOber} (kurz ,,GUI``) bedient. Der Anwender soll in der Lage, sein bereits vorhandene Computerkenntnisse zu nutzen.

\subsection{Eingabegeräte}

Die \gls{GrafBenOber} soll Maus und Tastatureingaben unterstützen. Außerdem soll eine Eingabe per \gls{dragdrop} möglich sein.
Die Verwendung anderer Eingabegeräte ist nicht vorgesehen.

\subsection{Überblick}

\begin{figure}[htbp]
	\begin{center}
		\includegraphics[width = 11cm]{Grafik/Konkreter-Anwendungsfall}
		\caption{Der Grundlegende Aufbau der Hauptbenutzeroberfläche}
		\label{GUI_Grundlage}
	\end{center}
\end{figure}

Abbildung \ref{GUI_Grundlage} zeigt den Aufbau einer möglichen \gls{GrafBenOber}. Dabei geben die Zahlen in den roten Kästchen auf der rechten Seite die logische Unterteilung an: Systemmenüleiste (1), Auswahl (2), Konfigurationsfeld (3)

Das Design der einzelnen Elemente und deren Layout kann sich im Verlauf der Anwendungsentwicklung ändern. Die grundlegenden Bausteine der \gls{GrafBenOber} werden sich in ihrer Aufgabe nicht ändern. Zu ihnen gehören:

\begin{itemize}
	\item \glspl{sensor}
	\item \glspl{transformation}
	\item \glspl{darstellung}
\end{itemize}

\subsection{Element Beschreibungen}

\subsubsection{Systemmenüleiste}

In der Systemmenüleiste befinden sich Funktionen, die Einfluss auf die gesamte \gls{GrafBenOber} haben.

\begin{tabular}[t]{p{1cm} p{10cm}} %Alle Systemmenübausteine in einer Tabelle
	\vspace{0cm}\includegraphics[width = 1 cm]{Grafik/PhyPiDAQ.png} & Der Name des Programmes wird hier angezeigt. Rechts daneben steht, sofern vorhanden, der Name der Datei, die gerade geöffnet ist.\newline\\
	\vspace{0cm}\includegraphics[width = 1 cm]{Grafik/Maximieren.png} & Das ,,Maximieren``-Symbol vergrößert das Programmfenster auf die maximale Größe. Die Größe ist von der Benutzungsumgebung abhängig.\newline\\
	\vspace{0cm}\includegraphics[width = 1 cm]{Grafik/Minimieren.png} & Das ,,Minimieren``-Symbol blendet das Programmfenster aus. Es ist weiterhin geöffnet und wieder aufrufbar. \\
	\vspace{0cm}\includegraphics[width = 1 cm]{Grafik/Schliessen.png} & Das ,,Schließen``-Symbol beendet die Anwendung. Vor dem Beenden findet eine Abfrage statt, ob der Anwender eventuell vorgenommene Änderungen speichern möchte.\\
\end{tabular}

\subsubsection{Auswahl}

In der Auswahl befinden sich Funktionen, die das Bearbeiten des Konfigurationsfeldes ermöglichen.

\begin{tabular}[t]{p{1cm} p{10cm}} %Auswahlelemente Teil 1
	\vspace{0cm}\includegraphics[width = 1 cm]{Grafik/Datei.png} & Der Benutzer hat die Möglichkeit Dateien zu bearbeiten. Die wichtigsten Funktionen sind:
	\begin{itemize} 
		\item Anlegen einer neuen Datei
		\item Speichern der aktuellen Datei
		\item Öffnen einer bereits erstellten Datei
	\end{itemize}\\
\end{tabular}

\begin{tabular}[t]{p{1cm} p{10cm}} %Auswahlelemente Teil 2
	\vspace{0cm}\includegraphics[width = 1 cm]{Grafik/Bearbeiten.png} & Der Benutzer hat die Möglichkeit, den Inhalt der aktuell geöffneten Datei zu bearbeiten. Die wichtigsten Funktionen sind:
	\begin{itemize} 
		\item Kopieren eines ausgewählten Objektes
		\item Einfügen eines gespeicherten Objektes
		\item Anpassen eines ausgewählten Elements. Diese Einstellungsmöglichkeiten sind von dem Objekt abhängig.
	\end{itemize}
\end{tabular}

\begin{tabular}[t]{p{1cm} p{10cm}} %Auswahlelemente Teil 3
	\vspace{0cm}\includegraphics[width = 1 cm]{Grafik/Sensor.png} & Der Benutzer kann sich eine Auswahl von \glspl{sensor} anzeigen lassen. Die \glspl{sensor} können anschließend in das Konfigurationsfeld eingefügt werden. Die angezeigte Auswahl ist unabhängig von den angeschlossenen Sensoren\newline
\end{tabular}

\begin{tabular}[t]{p{1cm} p{10cm}} %Auswahlelemente Teil 4
	\vspace{0cm}\includegraphics[width = 1 cm]{Grafik/Verbindung.png} & Der Benutzer kann sich eine Auswahl von \glspl{transformation} anzeigen lassen. Die \glspl{transformation} können anschließend per \gls{dragdrop} in das Konfigurationsfeld gezogen werden. Die Auswahl kann unter Optionen erweitert werden.\newline\\
	\vspace{0cm}\includegraphics[width = 1 cm]{Grafik/Darstellung.png} & Der Benutzer kann sich eine Auswahl von \glspl{darstellung} anzeigen lassen. Die \glspl{darstellung} können anschließend in das Konfigurationsfeld eingefügt werden. Die Auswahl kann erweitert werden.\newline
\end{tabular}

\begin{tabular}[t]{p{1cm} p{10cm}} %Auswahlelemente Teil 5
	\vspace{0cm}\includegraphics[width = 1 cm]{Grafik/Optionen.png} & Der Benutzer kann sich eine Auswahl an Optionen anzeigen lassen. Beispiele hierfür sind
	\begin{itemize} 
		\item das Ändern von angezeigten Farben
		\item das Erweitern der Transformations- und Darstellungsauswahl
	\end{itemize}\\
	\vspace{0cm}\includegraphics[width = 1 cm]{Grafik/Hilfe.png} & Der Benutzer kann sich eine allgemeine Hilfe anzeigen lassen. In der Hilfe werden die Funktionen der Anwendung erklärt.\newline
\end{tabular}

Im Laufe der Entwicklung kann es sich als sinnvoll herausstellen, weitere Optionen einzufügen bzw. bestehende Optionen zusammenzufassen.

\subsubsection{Konfigurationsfeld}

Im Konfigurationsfeld wird die in Bearbeitung befindliche Messkonfiguration des Anwenders grafisch dargestellt.

\begin{tabular}[t]{p{1cm} p{10cm}}
	\vspace{0cm}\includegraphics[width = 1 cm]{Grafik/Information.png} & Das ,,Informations``-Symbol zeigt nach Aktivierung weiterführende Informationen zu dem Element an, zu dem es gehört. So würde beispielsweise bei einer Transformation die Funktion angezeigt werde, die sie realisiert.\newline\\\cline{1-2}
	\vspace{0cm}\includegraphics[width = 1 cm]{Grafik/Sensorelement.png} & Repräsentation eines Sensors, den PhyPiDAQ ansteuern kann. Voraussetzung hierfür ist die Existenz einer \gls{konfigdata}.\newline\\
	\vspace{0cm}\includegraphics[width = 1 cm]{Grafik/Eingabe-Sensor.png} & Die zu einem Sensor gehörende Eingabe. Die Eingabe wird nur angezeigt, wenn auch die Kanäle angezeigt werden.\newline\\
	\vspace{0cm}\includegraphics[width = 1 cm]{Grafik/Ausgabe-Sensor.png} & Die zu einem Sensor gehörende Ausgabe. Jeder Sensor hat genau eine Ausgabe.\newline\\
	\vspace{0cm}\includegraphics[width = 1 cm]{Grafik/Kanal.png} & Der zu einem Sensor gehörende Kanal. Die Anzahl an angezeigten Kanälen hängt von dem Sensor ab. In den Optionen kann eingestellt werden, ob die Kanäle angezeigt werden.\newline\\\cline{1-2}
	\vspace{0cm}\includegraphics[width = 1 cm]{Grafik/Verbindungselement.png} & Repräsentation einer \gls{transformation}. Jedem Eingang können durch Einfügen einer Verbindung Daten übergeben werden.\newline\\
	\vspace{0cm}\includegraphics[width = 1 cm]{Grafik/Eingabe-Verbindung.png} & Die Eingabe einer Transformation. Die Anzahl der verfügbaren Eingaben hängt von der Transformation ab.\newline\\
	\vspace{0cm}\includegraphics[width = 1 cm]{Grafik/Ausgabe-Transformation.png} & Die Ausgabe einer Transformation. Die Anzahl der verfügbaren Ausgaben hängt von der Transformation ab.\newline\\\cline{1-2}

	\vspace{0cm}\includegraphics[width = 1 cm]{Grafik/Verbindungspfeil.png} & Eine Verbindung zwischen zwei Elementen. Sie besitzt eine Richtung. Daten ,,fließen`` also nicht in beide Richtungen.\newline\\\cline{1-2}
	\vspace{0cm}\includegraphics[width = 1 cm]{Grafik/Darstellungsfenster.png} & Eine Darstellung erhält Daten und stellt diese grafisch da.\newline\\

\end{tabular}

\clearpage

\subsection{Erweiterungsmöglichkeiten}

\subsubsection{Startbildschirm}

\begin{figure}[htbp]
	\begin{center}
		\includegraphics[width = 9cm]{Grafik/Startbildschirm.png}
		\caption{Ein Beispiel für einen Startbildschirm}
		\label{startbildschirm}
	\end{center}
\end{figure}

Eine Erweiterungsmöglichkeit wäre das Einfügen eines Startbildschirmes beim Öffnen der Anwendung. Dieser ist in Abbildung \ref{startbildschirm} angedeutet.

\subsubsection{Fehlermeldung}

Die Fehlermeldungen könnten in eigenen Fenstern erscheinen. Abbildung \ref{fehlerfenster} zeigt ein mögliches Design. Für unterschiedliche Fehler könnten unterschiedliche Designs verwendet werden.

\begin{figure}[htbp]
	\begin{center}
		\includegraphics[width = 9cm]{Grafik/Fehlerfenster}
		\caption{Ein Beispiel für ein Fehlermeldungsfenster}
		\label{fehlerfenster}
	\end{center}
\end{figure}

\clearpage
\section{Zeit- und Ressourcenplanung}\label{zeit}

\subsection{Projektphasen} \label{projektphasen}

\begin{tabular}{| l | l | l | r |}
	\hline
	\textbf{Phase} & \textbf{Verantwortlicher} & \textbf{Zeitraum} & \textbf{Kolloquium} \\ \hline
	Pflichtenheft & Jan Küblbeck & KW 20–22 & 04.06.2019 \\
	Entwurf & Leon Huck & KW 23–26 & 02.07.2019 \\
	Implementierung & Stefan Geretschläger & KW 27–29, 31 & 13.08.2019 \\
	Klausurenphase & — & KW 30, 32 & — \\
	Qualitätssicherung & David Gawron & KW 33–35 & 03.09.2019 \\
	Abnahme & — & KW 36 & — \\
	Abschlussprüfung & Linus Ruhnke & KW 37/38 & t.b.d. \\
	\hline
\end{tabular}

\subsection{Risikomanagement}

Beim Herstellungsprozess einer Software gibt es viele Risiken, die zu berücksichtigen sind. Wir haben bezüglich einiger Worst-Case-Szenarien die vermutete Auftrittswahrscheinlichkeit, die Kosten für Vorsichtsmaßnahmen und die möglichen negativen Auswirkungen auf \hyperref[funktionale]{Funktionalität}  und \hyperref[quali]{Qualität} unserer Anwendung gegeneinander abgewogen und uns damit für das folgende Risikomanagement entschieden. 

In der Planungsphase sticht vor allem das Risiko hervor, dass wir uns zu viel vornehmen und den resultierenden Aufwand in späteren Projektphasen, unterschätzen. Unsere Gegenmaßnahme dafür ist die Einteilung der Kriterien in Prioritätsklassen. Wir setzen nur die Features als Musskriterien, die unbedingt in der Software enthalten sein müssen. Der Rest ist als Soll- oder Wunschkriterien optional.
 
In der Entwurfsphase scheint das größte Risiko zu sein, dass bei den Integrationstests der Qualitätssicherungsphase unvorhergesehene Probleme auftauchen und unsere Entwürfe für dann notwendige Fall-Backs oder Work-Arounds zu unflexibel sind. Unsere Gegenmaßnahme dazu ist ein Fallback-Plan für wichtige Entscheidungen der Definitionsphase. Diese Pläne werden wir in der Entwurfsphase berücksichtigen. Dazu zählt z.B. der Fallbackplan, dass unsere Anwendung auch auf einem Desktop-Computer laufen kann, falls die Ressourcen des \gls{RasPi} nicht ausreichen.
 
In der Implementierungsphase ist das Hauptrisiko das Produzieren von Implementierungsfehlern. Unsere Gegenmaßnahme hierzu ist eine gute Testabdeckung, um das Risiko von Fehlern zu minimieren.
 
In der Qualitätssicherungsphase liegt das Hauptrisiko darin, dass wir uns zu sehr auf Kleinigkeiten konzentrieren und das Wesentliche 
übersehen. Dazu planen wir unsere Anwendung mit Schülern zu testen. Damit sollen wir zuvor übersehene Schwierigkeiten hingewiesen zu werden.

Ein Risiko, das sich durch alle Projektphasen durchzieht, ist der Ausfall eines Teammitglieds, beispielsweise durch Krankheit. Bei einem Team von fünf Mitgliedern, bedeutet ein Ausfall ein Verlust von 20\%. Unsere Gegenmaßnahme ist die Einplanung von Pufferzeit in jede Projektphase. Ein weiteres phasenübergreifendes Risiko ist ein Ausfall wegen Urlaub oder Klausuren. Dieses Risiko wird aber durch unsere frühzeitige Planung und durch das Einplanen von Klausurpausen minimiert.

\subsection{Spezielle Anforderungen}\label{entwicklungsumgebung}

Für die Umsetzung des Projekts ist Zugang zu funktionierender Hardware (konkret: mindestens ein \gls{RasPi}, möglichst mehrere \glspl{sensor}, Peripheriegeräte) notwendig, um die Funktionalität des Produkts sicherzustellen.\newline
Ansonsten muss gegebenenfalls auf simulierte \gls{messdaten} zurückgegriffen werden.

Das Projekt ist von \gls{PhyPiDAQ} abhängig, das heißt Funktionen welche in PhyPiDAQ fehlerhaft sind können möglicherweise nicht korrekt implementiert werden. PhyPiDAQ sollte beispielsweise im Fall eines Programmfehlers eindeutige Fehlercodes produzieren, damit der Benutzer aussagekräftig über das Problem informiert werden kann.\newline
Durch Kooperation mit den PhyPiDAQ-Entwicklern können derartige Probleme vermieden werden.

\clearpage
\section{Glossar}\label{glossar}

\renewcommand*{\glossarysection}[2][]{}	% prevents double glossary section heading
\printnoidxglossaries				% generate pdf twice when adding new entries


\end{document}\grid
