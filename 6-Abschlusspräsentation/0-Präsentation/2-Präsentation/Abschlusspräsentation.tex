\documentclass{beamer}

% Choose how your presentation looks.
\mode<presentation>
{
\usetheme{Darmstadt}      % or try Darmstadt, Madrid, Warsaw, ...
\usecolortheme{wolverine} % or try albatross, beaver, crane, ...
\usefonttheme{structurebold}  % or try serif, structurebold, ...
\setbeamertemplate{navigation symbols}{}
\setbeamertemplate{caption}[numbered]

%Customerize Prasentation
\definecolor{Grau}{HTML}{CCCCCC}
\definecolor{GrauDark}{HTML}{777777} % auf weißem hintergrund
\definecolor{Orange}{HTML}{EA5B10}
\setbeamercolor{palette primary}{bg=Grau}
\setbeamercolor{palette primary}{fg=Orange}
\setbeamercolor{normal text}{fg=GrauDark}
\setbeamercolor{structure}{fg=Orange} % farbe der items
\setbeamertemplate{itemize item}[circle]
\setbeamercolor{mini frame}{fg=Orange}
\setbeamercolor{section in head/foot}{bg=Grau}
\setbeamercolor{section in head/foot}{fg=Orange}
\setbeamercolor{subsection in head/foot}{bg=white}
\setbeamercolor{subsection in head/foot}{fg=GrauDark}
\setbeamercolor{headline}{bg=Grau}
\setbeamercolor{block body}{bg=white}
\setbeamercolor{frametitle}{bg=white}
}

\usepackage[utf8]{inputenc}
\usepackage[T1]{fontenc}
\usepackage{ae}
\usepackage{ngerman}
\usepackage{calc}
\usepackage{graphicx}
\usepackage{makecell}
\usepackage{pgfplots}
\usepackage[autostyle=true,german=quotes]{csquotes}
\pgfplotsset{compat=1.4}

\newenvironment<>{varblock}[2][.9\textwidth]{%
  \setlength{\textwidth}{#1}
  \begin{actionenv}#3%
    \def\insertblocktitle{#2}%
    \par%
    \usebeamertemplate{block begin}}
  {\par%
    \usebeamertemplate{block end}%
  \end{actionenv}}


\title{FreeJDAQ}
\subtitle{Visuelle Programmiersprache zur Datenerfassung auf
einem Raspberry Pi}
\author{David Gawron, Stefan Geretschlaeger, Leon Huck, \\ 
Jan Kublbeck, Linus Ruhnke }
\date{\today}

\begin{document}

\begin{frame}
\titlepage
\end{frame}

\section{Einleitung}

\begin{frame}{Einleitung}
\frametitle{Projektvorstellung}
		 \center
       	 \includegraphics[width=\textwidth / 2]{Grafiken/FreeJDAQ.png} \\
		\textbf{Free} \textbf{J}ava \textbf{D}ata \textbf{A}c\textbf{q}uisition
\end{frame}

\begin{frame}{Einleitung}
\frametitle{Abgrenzungen}
 Was unser Produkt nicht enthält:
        \begin{itemize}
            \item Direkte Ansprache der Sensoren. (PhyPiDAQ)
            \item Visuelle Repraesentation der Messkonfiguration
        \end{itemize}
\end{frame}

\section{Softwaretechnik}

\begin{frame}{Softwaretechnik}
\frametitle{Grundaufbau}
\begin{center}
       	 \includegraphics[width = 5cm]{Grafiken/DeploymentDiagram.png}
    	\end{center}
\end{frame}

\begin{frame}{Softwaretechnik}
\frametitle{Paketdiagramm}
\begin{figure}[htbp]
    	\begin{center}
       	 \includegraphics[width = 10cm]{Grafiken/Klassendiagramm.png}
    	\end{center}
	\end{figure}
\end{frame}

\section{Statistiken}

     \begin{frame}{Unit-Tests}
    \centering
    \begin{tikzpicture}
    \begin{axis}[
    ybar,
    enlarge x limits=0.2,
    width=0.95\textwidth,
    height=0.8\textheight,
    ylabel={Anzahl},
    xtick=data,
    nodes near coords,
    ymin=0,
    legend pos=north west,
    xticklabel style={rotate=30},
    symbolic x coords={GUI,Controller,FileService,Cache,Model,Backend},xticklabels={GUI,Controller,FileService,Cache,Model,Backend},
    ]
    
    \addplot coordinates
    {(GUI,1) (Controller,7) (FileService,12) (Cache,3) (Model,66) (Backend,18)};
    \end{axis}
    \end{tikzpicture}
    Insgesamt 107 Testcases, zzgl. 33 GUI - Klickstrecken
\end{frame}

\begin{frame}{Statistiken}
\frametitle{Testabdeckung}
  \centering
    \begin{tikzpicture}
    \begin{axis}[
    ybar,
    enlarge x limits=0.2,
    width=0.95\textwidth,
    height=0.8\textheight,
    ylabel={Prozent},
    xtick=data,
    nodes near coords,
    ymin=0,
    legend pos=north west,
    xticklabel style={rotate=30},
    symbolic x coords={Controller,FileService,GUI,Cache,Model,Backend},xticklabels={Controller,FileService,GUI,Cache,Model,Backend},
    ]
    
    \addplot coordinates
    {(Controller,53) (FileService,79) (GUI,84) (Cache,84) (Model,85) (Backend,96)};
    \end{axis}
    \end{tikzpicture}
    Insgesamt 80 Prozent
\end{frame}

\begin{frame}{Statistiken}
\frametitle{GitHub - FreeJDaq - Commits}
    \centering
    \begin{tikzpicture}
    \begin{axis}[
    ybar,
    enlarge x limits=0.2,
    width=0.95\textwidth,
    height=0.8\textheight,
    ylabel={Anzahl der Commits},
    xtick=data,
    nodes near coords,
    ymin=0,
    legend pos=north west,
    xticklabel style={rotate=45},
    symbolic x coords={1.7,8.7,15.7,22.7,29.7,5.8,12.8,19.8,26.8,1.9},xticklabels={1.7 (I),8.7,15.7,22.7,29.7,5.8 (Q),12.8,19.8,26.8,1.9 (IA)},
    ]
    
    \addplot coordinates
    {(1.7,6) (8.7,114) (15.7,123) (22.7,111) (29.7,205) (5.8,49) (12.8,72) (19.8,112) (26.8,29) (1.9,21)};
    \end{axis}
    \end{tikzpicture}
    Insgesamt 842 Commits, 54/64 Issues closed, (15.09, 18:00 Uhr)
\end{frame}

 \begin{frame}{Lines of Code}
\frametitle{GitHub - FreeJDaq - Lines of Code}
        \begin{table}[t]
            \begin{center}
                \begin{tabular}{ | l | c | }
                    \hline
                    Datei & Anzahl Zeilen \\
                    \hline
                    Java & 5552 \\
			Main & 4013 \\
			Test & 1539 \\ \Xhline{0.8pt}
                    Gesamt (inklusive Kommentar- und Leerzeilen) & 12776 \\ \hline
                \end{tabular}
            \end{center}
        \end{table}
	Verteilt über 122 Mainklassen und 23 Testklassen
    \end{frame}

\begin{frame}{Statistiken}
\frametitle{GitHub - DAQDocuments}
  \centering
    \begin{tikzpicture}
    \begin{axis}[
    ybar,
    enlarge x limits=0.2,
    width=0.95\textwidth,
    height=0.8\textheight,
    ylabel={Wörter},
    xtick=data,
    nodes near coords,
    ymin=0,
    legend pos=north west,
    xticklabel style={rotate=15},
    symbolic x coords={Pflichtenheft, Entwurf, Implementierung, Qualitätssicherung, Interne Abnahme},xticklabels={Pflichtenheft, Entwurf, Implementierung, Qualitätssicherung, Interne Abnahme},
    ]
    
    \addplot coordinates
    {(Pflichtenheft,7600) (Entwurf,8000) (Implementierung,4900) (Qualitätssicherung,5900) (Interne Abnahme,1404)};
    \end{axis}
    \end{tikzpicture}
    Insgesamt ca. 27800 Wörter über 994 Commits (15.09, 18:00 Uhr)
\end{frame}

\section{Tools}

\begin{frame}{Tools}
\frametitle{Allgemein}

 \begin{columns}
            \begin{column}{0.3\textwidth}
                \begin{block}{UML}
                    \center
                    \includegraphics[width=(\textwidth / 2)]{Grafiken/PlantUmlIcon.png} %TODO find better logo
                \end{block}
            \end{column}
            \begin{column}{0.3\textwidth}
                \begin{block}{Unit-Testing}
                    \center
                    \includegraphics[width=(\textwidth / 3)]{Grafiken/JUnitIcon.png}
                \end{block}
                \begin{block}{IDE}
                    \center
                    \includegraphics[width=\textwidth / 4]{Grafiken/EclipseIcon.png}
                \end{block}
		   \begin{block}{Yaml-Editor}
		   \center
		   \includegraphics[width=\textwidth / 2]{Grafiken/SnakeYamlIcon.png}
                \end{block}
		\begin{block}{SSH}
		   \center
		   \includegraphics[width=\textwidth / 4]{Grafiken/SSHJIcon.png}
                \end{block}
		
            \end{column}
                \begin{column}{0.3\textwidth}
                    \begin{block}{Build Management}
                        \includegraphics[width=(\textwidth)]{Grafiken/MavenIcon.png}
                    \end{block}
                    \begin{block}{Statische Codeanalyse}
                        \center
                        \includegraphics[width=(\textwidth / 4)]{Grafiken/JacocoIcon.png}
                        \includegraphics[width=(\textwidth / 2)]{Grafiken/EclEmmaIcon.png}
       	 	     \includegraphics[width=(\textwidth / 2)]{Grafiken/SonarlintIcon.png}
                    \end{block}
                \end{column}
        \end{columns}
    \end{frame}

\section{Lernefahrung}

\begin{frame}
\frametitle{Probleme}
\begin{itemize}
\item Teamkommunikation in den ersten Phasen 
\item Nacharbeiten von Fehlern oder Vervollständigung
\item Technologiewahl $\rightarrow$ Technologiewechsel
\end{itemize}
\end{frame}

\begin{frame}
\frametitle{Was haben wir gelernt}
\begin{itemize}
\item Phasen planen  $\rightarrow$ Meilensteine, Deadlines setzen und Zuständigkeiten zuteilen.
\item Arbeitsverteilung gleichmäßig über den Zeitraum verteilen.
\item Meilensteine überprüfen und ggf. Ressourcen verschieben.
\item Vor der Implementierung die nötigen Tools aussuchen und in diese einarbeiten.
\end{itemize}
\end{frame}

\section{Livedemo}
\begin{frame}[plain]
\center
{\huge Livedemo}
\end{frame}


\appendix
\begin{frame}[plain]
\frametitle{Quellen}
\begin{itemize}
\item https://github.com/osl2/DAQ-Documents
\item https://github.com/osl2/PhyPiDAQ
\item https://github.com/GuenterQuast/PhyPiDAQ
\item http://plantuml.com/de/
\item https://junit.org/junit5/
\item https://www.eclipse.org/ide/
\item https://bitbucket.org/asomov/snakeyaml/src
\item https://github.com/hierynomus/sshj
\item https://maven.apache.org/
\item https://www.eclemma.org/
\item https://www.eclemma.org/jacoco/
\item https://www.sonarlint.org/
\end{itemize}
\end{frame}

\end{document}