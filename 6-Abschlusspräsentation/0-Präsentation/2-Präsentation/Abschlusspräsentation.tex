\documentclass[aspectratio=169]{beamer}

% Choose how your presentation looks.
\mode<presentation>
{
    \usetheme{Darmstadt}      % or try Darmstadt, Madrid, Warsaw, ...
    \usecolortheme{wolverine} % or try albatross, beaver, crane, ...
    \usefonttheme{structurebold}  % or try serif, structurebold, ...
    \setbeamertemplate{navigation symbols}{}
    \setbeamertemplate{caption}[numbered]

%FrameNR
\setbeamertemplate{footline}{%
\begin{beamercolorbox}[wd=\paperwidth,ht=2.25ex,dp=1ex]{Grey}%
\hspace*{1ex} \insertframenumber{} / \inserttotalframenumber
\end{beamercolorbox}}%

    
    %Customerize Prasentation
    \definecolor{Grau}{HTML}{CCCCCC}
    \definecolor{GrauDark}{HTML}{777777} % auf weißem hintergrund
    \definecolor{Orange}{HTML}{EA5B10}
    \setbeamercolor{palette primary}{bg=Grau}
    \setbeamercolor{palette primary}{fg=Orange}
    \setbeamercolor{normal text}{fg=GrauDark}
    \setbeamercolor{structure}{fg=Orange} % farbe der items
    \setbeamertemplate{itemize item}[circle]
    \setbeamercolor{mini frame}{fg=Orange}
    \setbeamercolor{section in head/foot}{bg=Grau}
    \setbeamercolor{section in head/foot}{fg=Orange}
    \setbeamercolor{subsection in head/foot}{bg=white}
    \setbeamercolor{subsection in head/foot}{fg=GrauDark}
    \setbeamercolor{headline}{bg=Grau}
    \setbeamercolor{block body}{bg=white}
    \setbeamercolor{frametitle}{bg=white}
}

\usepackage[utf8]{inputenc}
\usepackage[T1]{fontenc}
\usepackage{ae}
\usepackage{ngerman}
\usepackage{calc}
\usepackage{graphicx}
\usepackage{makecell}
\usepackage{pgfplots}
\usepackage[autostyle=true,german=quotes]{csquotes}
\usepackage{url}
\pgfplotsset{compat=1.4}

\newenvironment<>{varblock}[2][.9\textwidth]{%
    \setlength{\textwidth}{#1}
    \begin{actionenv}#3%
        \def\insertblocktitle{#2}%
        \par%
        \usebeamertemplate{block begin}}
    {\par%
        \usebeamertemplate{block end}%
\end{actionenv}}


\title{FreeJDAQ}
\subtitle{Visuelle Programmiersprache zur Datenerfassung im Schulunterricht auf
    einem Raspberry Pi}
\author{David Gawron, Stefan Geretschlaeger, Leon Huck,\\
 \underline{Jan Kublbeck}, \underline{Linus Ruhnke}}
\date{23. September 2019}

\begin{document}
    
    \begin{frame}
        \titlepage
    \end{frame}
    
    \section{Einleitung}
    
%Problemstellungsfolie
\begin{frame}{Einleitung}
	\frametitle{Problemstellung}
	\center
\begin{figure}[htbp]
       \includegraphics[width=7cm]{Grafiken/Physik2.png} 
	\caption{Lieblingsfächer auf Grund des Interesses}
\end{figure}
\end{frame}

\begin{frame}{Einleitung}
\frametitle{Problemstellung}
\begin{figure}[htbp]
	\center
       \includegraphics[width=\textwidth]{Grafiken/Physik1.png}
	\caption{Interessen im Physikunterricht}
\end{figure}
\end{frame}
    
 %Allgemeine Folie mit der Zielbestimmung als Einstieg
    \begin{frame}{Einleitung}
        \frametitle{Projektvorstellung}
        \center
        \includegraphics[width=\textwidth]{Grafiken/Ziel_Der_Anwendung.png} \\
    \end{frame}




%Erweiterungsmöglichkeiten von PhyPiDAQ
\begin{frame}{Einleitung}
    \frametitle{Erweiterungsmöglickeiten von PhyPiDAQ}
    \framesubtitle{Von Prof. Dr. Günter Quast}
    \center
    \includegraphics[width=\textwidth-2cm]{Grafiken/Gegenueberstellung_PhyPiDAQ_FreeJDAQ.png} \\
\end{frame}

    %FreeJDAQ Logo
    \begin{frame}{Einleitung}
        \frametitle{Projektvorstellung}
        \center
        \includegraphics[width=5cm]{Grafiken/FreeJDAQ.png} \\
        \textbf{Free} \textbf{J}ava \textbf{D}ata \textbf{A}c\textbf{q}uisition
    \end{frame}


%GUI Erklärung
%\begin{frame}{Einleitung}
%    \frametitle{Funktionen des Konfigurations-Bereiches}
%    \center
%    \includegraphics[width=\textwidth]{Grafiken/Funktionen_Des_Konfigurations-Bereiches.png} \\
%\end{frame}

%GUI Erklärung
%\begin{frame}{Einleitung}
%    \frametitle{Funktionen des Messlauf-Bereiches}
%    \center
%    \includegraphics[width=\textwidth]{Grafiken/Messlauf-Bereich_Funktionen.png} \\
%\end{frame}

%GUI Erklärung
%\begin{frame}{Einleitung}
%    \frametitle{Funktionen des System-Bereiches}
%    \center
%    \includegraphics[width=\textwidth]{Grafiken/System-Bereich_Funktionen.png} \\
%\end{frame}

    \begin{frame}{Einleitung}
        \frametitle{Abgrenzungen}
        Was unser Produkt nicht enthält:
        \begin{itemize}
            \item Direkte Ansprache der Sensoren (PhyPiDAQ)
            \item Visuelle Repräsentation der Messkonfiguration
	     \item Abfangen von Fehlern beim Anschließen der Messtechnik
	     \item Erklärungen auf physikalischer Ebene
        \end{itemize}
    \end{frame}
    
    \section{Softwaretechnik}
    
    \begin{frame}{Softwaretechnik}
        \frametitle{Grundaufbau}
        \begin{center}
            \includegraphics[width =14cm]{Grafiken/DeploymentDiagram2.png}
        \end{center}
    \end{frame}
    
    \begin{frame}{Softwaretechnik}
        \frametitle{Paketdiagramm}
        \begin{figure}[htbp]
            \begin{center}
                \includegraphics[width = 11cm]{Grafiken/Klassendiagramm.png}
            \end{center}
        \end{figure}
    \end{frame}
    
    \section{Statistiken}
    
    \begin{frame}{Statistiken}
        \frametitle{GitHub - FreeJDaq - Lines of Code}
        \begin{table}[t]
            \begin{center}
                \begin{tabular}{ | l | c | }
                    \hline
                    \textbf{Datei} & \textbf{Anzahl Zeilen} \\
                    \hline
                    Anwendung & 4013 \\
                    Test & 1539 \\ \Xhline{0.8pt}
 			Gesamt & 5552 \\ \Xhline{0.8pt}
                    Gesamt (inklusive Kommentar- und Leerzeilen) & 12776 \\ \hline
                \end{tabular}
            \end{center}
        \end{table}
	\center
        Verteilt über 122 Mainklassen und 23 Testklassen. \\
	Insgesamt 846 Commits, 54/64 Issues closed.
    \end{frame}
    
    \begin{frame}{Unit-Tests}
        \centering
        \begin{tikzpicture}
        \begin{axis}[
        ybar,
        enlarge x limits=0.2,
        width=0.95\textwidth,
        height=0.8\textheight,
        ylabel={Anzahl},
        xtick=data,
        nodes near coords,
        ymin=0,
        legend pos=north west,
        xticklabel style={rotate=30},
        symbolic x coords={GUI,Controller,FileService,Cache,Model,Backend},xticklabels={GUI,Controller,FileService,Cache,Model,Backend},
        ]
        
        \addplot coordinates
        {(GUI,1) (Controller,7) (FileService,12) (Cache,3) (Model,66) (Backend,18)};
        \end{axis}
        \end{tikzpicture}
        Insgesamt 107 Testcases, zzgl. 33 GUI - Klickstrecken
    \end{frame}
    
    \begin{frame}{Statistiken}
        \frametitle{Testabdeckung}
        \centering
        \begin{tikzpicture}
        \begin{axis}[
        ybar,
        enlarge x limits=0.2,
        width=0.95\textwidth,
        height=0.8\textheight,
        ylabel={Prozent},
        xtick=data,
        nodes near coords,
        ymin=0,
        legend pos=north west,
        xticklabel style={rotate=30},
        symbolic x coords={Controller,FileService,GUI,Cache,Model,Backend},xticklabels={Controller,FileService,GUI,Cache,Model,Backend},
        ]
        
        \addplot coordinates
        {(Controller,53) (FileService,79) (GUI,84) (Cache,84) (Model,85) (Backend,96)};
        \end{axis}
        \end{tikzpicture}
        Insgesamt 80 Prozent Bedingungsüberdeckung.
       %Zweig-, Anwendungs-, einfache und mehrfache minimale Bedinungsüberdeckung
    \end{frame}
    
    \section{Tools}
    
    \begin{frame}{Tools}
        \frametitle{Allgemein}
        
        \begin{columns}
            \begin{column}{0.3\textwidth}
                \begin{block}{UML}
                    \center
                    \includegraphics[width=(\textwidth / 2)]{Grafiken/PlantUmlIcon.png} %TODO find better logo
                \end{block}
            \end{column}
            \begin{column}{0.3\textwidth}
                \begin{block}{Unit-Testing}
                    \center
                    \includegraphics[width=(\textwidth / 3)]{Grafiken/JUnitIcon.png}
                \end{block}
                \begin{block}{IDE}
                    \center
                    \includegraphics[width=\textwidth / 4]{Grafiken/EclipseIcon.png}
                \end{block}
		\begin{block}{Java-Bibliotheken}
			\center
                    \includegraphics[width=3cm]{Grafiken/SnakeYamlIcon.png}
                    \includegraphics[width=3cm]{Grafiken/JFreeChartIcon.png}
                    \includegraphics[width=(\textwidth / 4)]{Grafiken/SSHJIcon.png}
                \end{block}
                
            \end{column}
            \begin{column}{0.3\textwidth}
                \begin{block}{Build Management}
                    \includegraphics[width=(\textwidth)]{Grafiken/MavenIcon.png}
                \end{block}
                \begin{block}{Statische Codeanalyse}
                    \center
                    \includegraphics[width=(\textwidth / 4)]{Grafiken/JacocoIcon.png}
                    \includegraphics[width=(\textwidth / 2)]{Grafiken/EclEmmaIcon.png}
                    \includegraphics[width=(\textwidth / 2)]{Grafiken/SonarlintIcon.png}
                \end{block}
            \end{column}
        \end{columns}
    \end{frame}
    
    \section{Lernerfahrung}
    
    \begin{frame}
        \frametitle{Probleme}
        \begin{itemize}
            \item Teamkommunikation in den ersten Phasen 
            \item Nacharbeiten von Fehlern oder Vervollständigung
            \item Technologiewahl $\rightarrow$ Technologiewechsel
        \end{itemize}
    \end{frame}
    
    \begin{frame}
        \frametitle{Was haben wir gelernt}
        \begin{itemize}
            %Projektmanagement im allgemeinen
            \item Phasen planen  $\rightarrow$ Meilensteine, Deadlines setzen und Zuständigkeiten zuteilen
            \item Arbeitsverteilung gleichmäßig über den Zeitraum verteilen
            \item Meilensteine überprüfen und ggf. Ressourcen verschieben
            %Software-Entwicklung spezifisch
            \item Vor der Implementierung die nötigen Tools aussuchen und in diese einarbeiten
        \end{itemize}
	 \center
	 \includegraphics[width=8cm]{Grafiken/FreeJDAQCommits.png}
    \end{frame}
    
    \section{Livedemo}
    \begin{frame}
        \center
        {\huge Livedemo}
    \end{frame}

 \begin{frame}
\frametitle{Begriffserklärung}
     \textbf{Messkonfiguration:} \\
Virtueller Versuchsaufbau im Yaml-Format bestehend aus:

\begin{itemize}
\item Bausteinen (Ein- und Ausgänge)
\item Verbindungen
\end{itemize}
\textbf{Bausteine:}
\begin{itemize}
\item Sensoren
\item Transformationen (Funktion zur Veränderung der Messdaten)
\item Repräsentationen (Darstellung der Messdaten)
\end{itemize}
\textbf{Verbindungen:}
\begin{itemize}
\item Verknüpfungen zwischen den Ein- und Ausgänge der Bausteine.
\end{itemize}
    \end{frame}


\appendix
\begin{frame}
        \frametitle{Zusammenfassung}
\textbf{Zur Anwendung:}
\begin{itemize}
	\item Es wurde mit FreeJDAQ eine Basis geschaffen, welche Schülern und Physikinteressierten Menschen eine Plattform bietet, Messläufe einfach und schnell durchzuführen
	\item Weitere Produkteigenschaften und Erweiterungen können dieser Basis hinzugefügt werden
\end{itemize}
\textbf{Zur Gruppenarbeit:}
\begin{itemize}
	\item Trotz Schwierigkeiten während jeder Phase hat sich unsere Gruppendynamik positiv entwickelt
	\item Gewinnung wichtiger Erfahrung in der Projektplanung und Softwareentwicklung
        \end{itemize}
    \end{frame}

	
    \appendix
    \begin{frame}[plain]
	\center
	\huge{Vielen Dank für Ihre Aufmerksamkeit \\}

        \includegraphics[width=5cm]{Grafiken/FreeJDAQ.png} \\
        \textbf{Free} \textbf{J}ava \textbf{D}ata \textbf{A}c\textbf{q}uisition
    \end{frame}
    
    \appendix
    \begin{frame}[plain]
        \frametitle{Quellen}
        \begin{itemize}
            \item https://github.com/osl2/DAQ-Documents
            \item https://github.com/osl2/PhyPiDAQ
            \item https://github.com/GuenterQuast/PhyPiDAQ
            \item http://plantuml.com/de/
            \item https://junit.org/junit5/
            \item https://www.eclipse.org/ide/
            \item https://bitbucket.org/asomov/snakeyaml/src
            \item https://github.com/hierynomus/sshj
            \item https://maven.apache.org/
            \item https://www.eclemma.org/
            \item https://www.eclemma.org/jacoco/
            \item https://www.sonarlint.org/
	     \item http://www.jfree.org/jfreechart/
        \end{itemize}
    \end{frame}

	\appendix
    \begin{frame}[plain]
        \frametitle{Quellen}
\begin{itemize}
	\item Abbildung 1: \url{https://ag4physik.files.wordpress.com/2017/03/interessensforschung_strahl.pdf} \\
	\item Abbildung 2: \url {http://www.physikdidaktik.info/data/_uploaded/Delta_Phi_B/2015/Caglar-Oeztuerk(2015)Interessenforschung_DeltaPhiB.pdf}
\end{itemize}
    \end{frame}
    
\end{document}