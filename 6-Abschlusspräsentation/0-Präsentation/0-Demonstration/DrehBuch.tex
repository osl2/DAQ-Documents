\documentclass[parskip=full]{scrartcl}

\usepackage[utf8]{inputenc}			% Umlaute, Sonderzeichen
\usepackage[ngerman]{babel}			% deutsche Sprache
\usepackage{enumitem}				% Listen
\usepackage{graphicx}				% Grafiken
\usepackage{hyperref}				% Hyperlinks
\usepackage[nonumberlist]{glossaries}		% Glossar
\usepackage{amsmath}
\usepackage{pdfpages}				% PDF einbinden




\subject{Drehbuch Live Demonstration}
\title{Definition und Durchführung~von Messwertverarbeitung für~den~Physikunterricht auf~Basis~eines~Raspberry~Pis}
\subtitle{Version 1.0.0}
\author{Linus Ruhnke}
\date{\today}


\begin{document}

\maketitle
\clearpage

In diesem Dokument werden die einzelnen Aktionen während der Live Demonstration unserer Anwendung beschrieben.

\section{Live Demonstration}

\begin{itemize}

\item Die Anwendung wird durch Doppelklick auf die Fat-Jar geöffnet.
\item Nach Öffnen der Anwendung werden die einzelnen Elemente der Graphischen Oberfläche gezeigt und erklärt.
\item[1.] Das Konfigurationsfeld
\item[2.] Das Bausteinmenü und die Bausteineigenschaften
\item[3.] Optionen
\item[4.] Hilfe
\item[5.] Messdatenfenster und Visualisierung
\item[6.] Buttons
\item Eine Messkonfiguration wird durch Auswählen einer Konfigurationsdatei geöffnet.
\item Die einzelnen Teile der Messkonfiguration werden kurz beschrieben und erklärt.
\item Die Messkonfiguration wird gecheckt und es wird eine Erklärung zum "Check" gegeben.
\item Der Messlauf wird gestartet.
\item Die Messkonfiguration wird pausiert und fortgesetzt.
\item Die visuelle Darstellung wird gezeigt.
\item Die Messdaten werden gespeichert.
\item Eine Transformation in der Messkonfiguration wird verändert 
\item Der Messlauf wird erneut gestartet.
\item Die Unterschiede in den Messdaten werden erklärt.

\end{itemize}


\renewcommand*{\glossarysection}[2][]{}	% prevents double glossary section heading
\printnoidxglossaries				% generate pdf twice when adding new entries

\end{document}\grid
