\documentclass[parskip=full]{scrartcl}

\usepackage[utf8]{inputenc}			% Umlaute, Sonderzeichen
\usepackage[ngerman]{babel}			% deutsche Sprache
\usepackage{enumitem}				% Listen
\usepackage{graphicx}				% Grafiken
\usepackage{hyperref}				% Hyperlinks
\usepackage[nonumberlist]{glossaries}		% Glossar
\usepackage{amsmath}
\usepackage{pdfpages}				% PDF einbinden




\subject{Drehbuch Live Demonstration}
\title{Definition und Durchführung~von Messwertverarbeitung für~den~Physikunterricht auf~Basis~eines~Raspberry~Pis}
\subtitle{Version 1.0.0}
\author{Linus Ruhnke}
\date{\today}


\begin{document}

\maketitle


In diesem Dokument werden die einzelnen Aktionen während der Live Demonstration unserer Anwendung beschrieben.

\section{Live Demonstration}

Die Live Demonstration unserer Anwendung kann aufgrund längerer Ladezeiten oder Probleme beim Bedienen der Anwendung zu Verzögerungen kommen. Daher besteht unsere Live Demonstration aus dem Vorführen von 3 verschiedenen Messkonfigurationen. Falls es zeitlich knapp wird kann eine Messkonfiguration aus Zeitgründen ausgelassen werden.
Im Falle, dass unsere Anwendung Probleme hat, welche die Live Demonstration nicht möglich machen würde, gibt es einen Backup-Plan, dass die Demonstration als Video vorgespielt wird und kommentiert wird.

\begin{table}[t]
\begin{center}
\begin{tabular}{ | p{7cm} | p{7cm} | p{1cm} | }
\hline
\textbf{Aktion des Präsentierenden} & \textbf{Aktion des Unterstützenden} & \textbf{Zeit} \\
\hline
Erklärung der Verbindung zwischen FreeJDAQ und Raspberry Pi & Öffnen der Anwendung & 5s \\ \hline
Erklärung des Messaufbaus & Hochheben der einzelnen Elemente & 5s\\ \hline
Erklärung der graphischen Oberfläche & Öffnen des Bausteinmenüs und Eigenschaften &  20s \\ \hline
Erklärung der Konfigurationssprache und der spezifischen Konfiguration & Laden der ersten Konfiguration 			                     (demoConfig18B20.yaml) & 30s \\ \hline
Erklärung der textuellen Ausgabe & Starten der Anwendung & 5s \\  \hline
Erklärung der graphischen Ausgabe & - & 5s \\ \hline
Beschreibung der Messwertveränderung & Veränderung der Temperatur durch Anfassen des Sensors mit der Hand & 20s \\ \hline
Erklärung der weiteren Funktionen der Anwendung & Speichern des Messbildes und der Messdaten & 10s \\ \hline
&& \textbf{ca. 2min}  \\ \hline 
Erklärung zur zweiten Messkonfiguration & Laden der zweiten Messkonfiguration (demoConfigMMA8451.yaml) & 20s\\ \hline
Erklärung der textuellen Ausgabe & Starten der Anwendung & 10s \\  \hline
Beschreibung der Veränderungen während des Messlaufs & Bewegen des Beschleunigungssensor & 30s \\ \hline
&& \textbf{ca. 1min} \\ \hline
Erklärung zur dritten Messkonfiguration & Laden der zweiten Messkonfiguration (demoConfigAdderCombined.yaml) & 20s \\ \hline
Erklärung zu Transformationen & - & 20 s\\ \hline
Erklärung der textuellen Ausgabe & Starten der Anwendung & 10 s\\  \hline
Erklärung der graphischen Ausgabe & - & 10 s\\ \hline
&& \textbf{ca. 1min} \\ \hline
Beschreibung der Erweiterbarkeitsmöglichkeiten und weitere Funktionalitäten & - &\\ \hline

                \end{tabular}
            \end{center}
        \end{table}


\renewcommand*{\glossarysection}[2][]{}	% prevents double glossary section heading
\printnoidxglossaries				% generate pdf twice when adding new entries

\end{document}\grid
